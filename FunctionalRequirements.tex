
%% Title Page 
\title{\Huge Functional Requirements Document Spesification \\ 
	 Project: \\ 
	Cafeteria Management System: Reslove}
\author{
         \underline{T-RISE}\\
          Rendani Dau (13381467) \\
	Elana Kuun (12029522) \\
	Semaka Malapane (13081129) \\
	Antonia Michael (13014171) \\
	Isabel Nel (13070305)}

\date{\today}

\documentclass[12pt]{article}

\begin{document}
\maketitle
\break

%% Make table of contents
\tableofcontents
\break

%%now begin document

%%---------------------------------  INTRODUCTION -------------------------------------------
\section{Introduction}
This document contains the functional requirements specification for the Resolve Cafeteria Management System that will be created for Software Engineering (COS 301) at the University of Pretoria 2015, by the group T-RISE. In this document we will thoroughly discuss and layout the project's architecture requirements , functional requirements and application design to provide a clear view of the system as a whole. 

%% ------------------------------ VISION ------------------------------------------------------
\section{Vision}
The vision of this project is to implement a flexible, pluggable, fully functional software application that will be maintainable, with detailed supporting documentation and an instruction manual for the Cafeteria Management System. This system will then assist in the collection of payments for the cafeteria, manage inventory/stock, facilitate payments for access cards (or the use of unique access card numbers), and facilitate ordering from the cafeteria. The system will allow the cafeteria to use the system that they are currently using as it is with combination of a user friendly application and online facility to place orders and check stock and make predictions of needed stock for the following week.

%%---------------------------------- BACKGROUND -----------------------------------------
\section{Background}
As specified in the tender proposal document from Resolve - the cafeteria is currently cash only and does not accept bank cards or electronic payments. This makes it difficult for employees as they have to carry cash if they want to purchase anything from the cafeteria. In this case the employee might as well go to an outside food provider and
pay with their preferred method of payment. This problem wastes fuel for the employees, time for the company, and does not bring in the maximum amount of income to the cafeteria, hindering its growth and improvement.\\

Resolve is looking for a way to accept payments from employees for the canteen
using their employee access cards with an amount being deducted from their salary at the end of the month.\\

Resolve proposed the Cafeteria Management System to assist with this problem.
At our first meeting with Resolve they have also brought under our attention that at times the cafeteria does not have enough stock to make some of the menu items, thus the reporting of inventory or stock will also be part of the system. The system will predict what inventory/stock needs to be bought for the next week.

%% --------------------------------- ARCHITECTURE REQUIREMENTS ---------------------------------
\section{Architecture Requirementss}
The software architecture requirements include the access and integration requirements, quality
requirements, and architectural constraints. These points will be thouroughly discussed below under.

%% ---------- ACCESS CHANNEL REQUIREMENTS ------------
\subsection{Access Channel Requirements}
The Cafeteria Management System would be accessed by different users via  the online web page or through the mobile application that will be suited for different platforms. Different services will be available to different users. There are five types of users: Super User, Cafeteria Manager, Casher, Normal User, and Resolve Admin. \\

\paragraph{Super User\\}
The super user will be the only administrative user that will have global access to all the functionality of the Cafeteria Management System, in particular the Super User will have access to the branding of the Cafeteria Management system (changing the logo and so forth) . The super user will also have access to all the functionality of all the other users listed below.

\paragraph{ Cafeteria Manager\\}
The cafeteria manager will have the ability to view his/her own profile, edit his/her profile, and place orders as normal users can. This user will also be able to add and edit menu items, view the orders placed and the inventory or stock, and add or remove inventory or stock. 

\paragraph{ Cashier\\}
The cashier will be able to view his/her profile, edit his/her profile, view the orders placed, and tick off those orders that are done and collected. The cashier will also be able to make a purchase and check inventory or stock and add or remove inventory or stock, since some stock could have gotten old or rotten and needs to be removed from the available stock list. 

\paragraph{ Normal User\\}
The normal user will typically be a resolve employee registered on the Cafeteria Management System.  A normal user will only be able to view his/her profile, edit his/her profile, place orders, check if their order is ready, and print their balance reports.

\paragraph{ Resolve Admin\\}
The resolve admin user will only be able to view all the registered users and their total balance outstanding, this is for administrative and financial usage purposes requested by the resolve team.



%% ---------  QUILITY REQUIREMENTS -----------------------
\subsection{Quality Requirements}

\paragraph{ Preformance\\}
The Performance of a software system will be measured in the run time efficiency.  In the Cafeteria Management System we will be implementing technologies such as AngularJS wich will ensure for fast interactive services on the client side, giving the user a fast and effective way to order their meals from the Cafeteria without wasting voluble work time of the company.  Although the performance is also influenced by the architectural design we will ensure that processes on the server side are also fast and efficient to work smoothly with the client side. 

\paragraph{ Reliability\\}
When creating a software system it is not possible at the first run to create a system that is completely 'bug free', but a certain level of debugging and reliability of a system is needed to have it fully functional. Thus in the case of reliability unit testing is of upmost importance, if all pieces of code that gets added into the working system if fully tested for every possible scenario your system is more likely to have a very higher reliability than systems where only a few unit tests were conducted. When creating a system for a client it is important to make it as reliable as possible to promote good and satisfactory services as promised.

\paragraph{Scalability\\}
Scalability refers to a software system's ability to handle increased workloads. The Cafeteria Management System will be scalable if it can handle more than the currently registered employees, or even twice or three times  as many users.  

\paragraph{Security\\}
Security is considered as an important quality requirement in any online software system. For the Cafeteria Management System no user will be able to log into the system without being registered to the system. On registration all personal details, such as the employees e-mail address, will be verified to ensure all registered users can contacted if needed.

\paragraph{Flexability\\}
In the creation of the Cafeteria Management System it is important to keep the software as technology neutral as possible, this is why in the creation of an application of our online system the application will be able to work on multiple platforms facilitating a wide variety of users and the online facility will be able to open on all standard browsers. 

\paragraph{Maintainability\\}
Maintainability refers to the design of the system that needs to allow for the addition of new requirements without the risk of introducing new errors. In the process of implementing the Cafeteria Management System it is important to remember that the owner of the system might  want to add some functionality to the existing software at a later stage. It is thus important to implement coding standards to keep the software neat and readable. It will ensure that when the software is altered there will not be any trouble reading it or discovering bugs that were not for. Maintainability thus also refers to the testability of a software system - it is important to ensure that the software system is adequately tested at all levels.

\paragraph{Integrability\\}
Integrability  refers to the testing of separately developed components to ensure that they work together. As we make our different  modules using angular we will make sure each module when needed to interact with another model can do so efficiently. This will be achieved through unit testing and integration testing of the different modules as we build our system. This also means that if a module is removed from the system, the system will be able to run smoothly, it will not disrupt the whole system. If we thus modify one of the modules it won't disrupt the rest of the system. 

\paragraph{Usability\\}
Usability can be considered as a core quality requirement . \\
Usability involves measuring the user's performance with regard to the software system.  It is important to have a usable and pluggable system that the staff members of Resolve can use with ease; this implies that the site does not break down every time you click a link for example. \\
Usability of the Cafeteria Management System will be ensured by unit testing, all aspects of each module of our system will be thoroughly  tested before it will get passed on to be implemented in the working system .



%% ----------- INTEGRATION REQUIREMENTS ---------------
\subsection{Integrationl Requirements}

%% ----------ARCHITECTURE CONSTRAINTS----------------
\subsection{Architecture Constraints}
Technologies we will be using in the creation of the Cafeteria Management System includes the following: 

\begin{itemize}
  \item HTML - The Software system will be mainly web-based.

  \item JavaScript together with AngularJS - this will enable us to add extra functionality to our web page and modulirise the 	 	system thus also helping us to implement dependancy injection.

 \item CSS together with BootStrap - which will allow us to style our page and also make it interactive.

 \item PHP togerther with mySQL - this will be used to handle the database on the server side.

 \item Apatche Server - we will be hosting our data base on an Apatche server which is open scource

 \item Phone gap will be used to convert our web page into a usable application which will then look like the online webpage but canwill be acceessable from multiple platforms.

\end{itemize}
The above mentioned thecnologies will be our basis we will create our system on, but as we are busy building the Cafeteria Management System we will add other thechnologies as needed. 


%%--------------------------------------- FUNCTIONAL REQUIREMENTS--------------------------------
\section{Functiona Requirements and Aplication Design}

%% ---------------USE CASE PRIORITIZATION -------------------
\subsection{Use Case Prioritization}

%%--------------- USE CASE/SERVICE CONTRACTS --------------
\subsection{Use Case/Service Contracts}

%% -------------- REQUIRED FUNCTIONALITY --------------------
\subsection{Required Functionality}

%% --------------PROSESS SPESIFICATION ------------------------
\subsection{Process Spesification}

%%--------------- DOMAIN MODEL ---------------------------------
\subsection{Domain Model}


\section{Open Issues}








\end{document}

