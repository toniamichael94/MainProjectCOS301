
%% Title Page 
\title{\Huge Functional Requirements Document Spesification \\ 
	 Project: \\ 
	Cafeteria Management System: Reslove}
\author{
         \underline{T-RISE}\\
          Rendani Dau (13381467) \\
	Elana Kuun (12029522) \\
	Semaka Malapane (13081129) \\
	Antonia Micheal (13014171) \\
	Isabel Nel (13070305)}

\date{\today}

\documentclass[12pt]{article}

\begin{document}
\maketitle
\break

%% Make table of contents
\tableofcontents
\break

%%now begin document

%%---------------------------------  INTRODUCTION -------------------------------------------
\section{Introduction}
This Document contains the Functional Requirements Specification for the Resolve Cafeteria Management System that will be created for Software Engineering 301 at the University of Pretoria 2015, by the group T-RISE. In this document we will thoroughly discuss and layout the project's architecture requirements , functional requirements and application design to give a clear view of the system as a whole . 

%% ------------------------------ VISION ------------------------------------------------------
\section{Vision}
The vision of this project is to fully implement a flexible, pluggable, fully functional software application that will be maintainable, with detailed supporting documentation and an instruction manual for the Cafeteria Management System. This system will then assist in the collection of payments for the cafeteria, manage inventory/stock, facilitate payments for access cards (or the use of unique access card numbers) and facilitate ordering from the cafeteria. The system will still allow the cafeteria to use the system they are using currently as it is with combination of a user friendly application and online facility to place orders and check stock and make predictions of needed stock for the following week.

%%---------------------------------- BACKGROUND -----------------------------------------
\section{Background}
As specified in the tender proposal document from Resolve - the cafeteria is currently cash only and does not accept bank cards or electronic payments. This makes it difficult for employees as they have to carry cash. In this case the employee might as well go to an outside food provider and
pay with their preferred method of payment. This problem wastes fuel for the employees, time for the company and does not bring in the maximum amount of income to the cafeteria, hindering its growth and improvement.\\

Resolve is looking for a way to accept payments from employees for the canteen
using their employee access cards, with an amount being deducted from their salary at the end of the month.\\

Resolve Proposed the Cafeteria Management System to assist with this problem.
At our first meeting with Resolve, they have also brought under our attention that at times the cafeteria does not have enough stock to make some of the menu items, and thus the reporting of inventory or stock will also be part of this solution system containing predictions on what inventory/stock needs to be bought for the next week.

%% --------------------------------- ARCHITECTURE REQUIREMENTS ---------------------------------
\section{Architecture Requirementss}
The software architecture requirements include the access and integration requirements, quality
requirements and architectural constraints, these points will be thouroughly descussed below under respective headings.

%% ---------- ACCESS CHANNEL REQUIREMENTS ------------
\subsection{Access Channel Requirements}
The Cafeteria Management System would be accessed by different users via  the online web-page or through the mobile application that will be suited for different platforms . The following services would be available to the respective users (Super User, Cafeteria Manager, Casher, Normal User, Resolve Admin) as listed below: \\

\paragraph{Super User\\}
The Super User will be the only administrative user that will have global access to all the functionality of the Cafeteria Management System, in particular the Super User will have access to branding the Cafeteria Management system (changing the logo and so forth) . The Super User will also have access to all the functionality of all the other users listed below.

\paragraph{ Cafeteria Manager\\}
The Cafeteria Manager will have the ability to view his/her own profile, edit his/her profile,   place orders as normal users can, but he/she will also be able to add menu items and edit menu items , the manager will also be able to view the orders placed and the inventory or stock, add inventory or stock and remove inventory or stock.

\paragraph{ Casher\\}
The Casher will be able to view his/her profile, edit his/her profile,  view the orders placed, tick off the orders that is done and collected. The casher will also be able to make a purchase and check inventory or stock, add inventory or stock and remove inventory or stock, since some stock could have gotten old or rotten and needs to be removed from the available stock list. 

\paragraph{ Normal User\\}
The Normal User will be typically a resolve employee registered on the Cafeteria Management System.  A normal user will only be able to view his/her profile, edit his/her profile, place orders, check if order is ready and print balance reports.

\paragraph{ Resolve Admin\\}
The Resolve Admin user will only be able to view all the registered users and their total balance outstanding, this is for administrative and financial usage purposes requested by the resolve team.




%% ---------  QUILITY REQUIREMENTS -----------------------
\subsection{Quality Requirements}

%% ----------- INTEGRATION REQUIREMENTS ---------------
\subsection{Integrationl Requirements}

%% ----------ARCHITECTURE CONSTRAINTS----------------
\subsection{Architecture constraints}

%%--------------------------------------- FUNCTIONAL REQUIREMENTS--------------------------------
\section{Functiona Requirements and Aplication Design}

%% ---------------USE CASE PRIORITIZATION -------------------
\subsection{Use Case Prioritization}

%%--------------- USE CASE/SERVICE CONTRACTS --------------
\subsection{Use Case/Service Contracts}

%% -------------- REQUIRED FUNCTIONALITY --------------------
\subsection{Required Functionality}

%% --------------PROSESS SPESIFICATION ------------------------
\subsection{Process Spesification}

%%--------------- DOMAIN MODEL ---------------------------------
\subsection{Domain Model}


\section{Open Issues}








\end{document}

