\documentclass[a4paper,12pt]{article}
\usepackage[hidelinks]{hyperref}
\usepackage{graphicx}
\usepackage{float}
\usepackage{caption}
\usepackage{array}
\usepackage{tabu}
%% Title Page 
\title{ \rule{\textwidth}{1pt}  \\ \Huge User Manual \\ 
	\Large Cafeteria Management System: Resolve Solution Partners (Pty) Limited \\
	\small Client: Gareth Botha and Jaco Pieterse}
\author{
         \underline{T-RISE}\\
          Rendani Dau (13381467) \\
	Elana Kuun (12029522) \\
	Semaka Malapane (13081129) \\
	Antonia Michael (13014171) \\
	Isabel Nel (13070305)}

\date{\today \\ \rule{\textwidth}{1pt}}
%\documentclass[12pt]{article}

\begin{document}
\maketitle
\break

%% Make table of contents
\tableofcontents
\break

%%now begin document

 \begin{tabu} to 0.8\textwidth { | X[l] | X[l] | }
 \hline
 \textbf{Document Title} & Testing Documentation \\
 \hline
 \textbf{Document Identification}  & Document 0.0.6 \\
 \hline
 \textbf{Author}  & Rendani Dau, Isabel Nel, Elana Kuun, Semaka Malapane, Antonia Michael \\
 \hline
 \textbf{Version} & 0.0.6\\
 \hline
 \textbf{Document Status} & Sixth Version - contains added tests for menu items controller and reporting as well as updates to other tests  \\
 \hline
 \end{tabu}

\begin{table}[h!]
\centering
 \begin{tabular}{||c c c c||} 
 \hline
 \textbf{Version} & \textbf{Date} & \textbf{Summary} & \textbf{Authors} \\ [0.5ex] 
 \hline\hline
 0.0.1 & 29 May 2015 &  First draft contains first two use cases  & Rendani Dau, \\ & & & Elana Kuun, \\ & & & Semaka Malapane, \\ & & & Antonia Michael \\ & & & Isabel Nel, \\ & & & \\
 \hline 
 & & & \\
 0.0.2 & 9 July 2015 &  Second draft adding Testing for  & Rendani Dau, \\ & & client side controllers & Elana Kuun, \\ & & for settings, password, superuser & Semaka Malapane, \\ & & and authentication &  Antonia Michael \\ & & & Isabel Nel \\   [1ex] 
\ & & & \\
 \hline 
 & & & \\
 0.0.3 & 20 July 2015 &  Third draft adding Testing for  & Rendani Dau, \\ & & models and controllers & Elana Kuun, \\ & & for inventory, placedOrders & Semaka Malapane, \\ & & register and authentication &  Antonia Michael \\ & & & Isabel Nel \\   [1ex] 
 & & & \\
 \hline 
 & & & \\
 0.0.4 & 23 July 2015 &  Fourth draft adding Testing for  & Rendani Dau, \\ & & models and controllers & Elana Kuun, \\ & & for manage cafeteria & Semaka Malapane, \\ & & and manage system &  Antonia Michael \\ & & & Isabel Nel \\   [1ex] 
& & & \\
 \hline 
 & & & \\
 0.0.5 & 25 August 2015 &  Fifth draft adding Testing for  & Rendani Dau, \\ & & orders, finance and & Elana Kuun, \\ & & cashier controllers& Semaka Malapane, \\ & & and updated spec &  Antonia Michael \\ & & & Isabel Nel \\   [1ex] 
& & & \\
 \hline 
 & & & \\
 0.0.6 & 24 September 2015 &  Sixth draft adding Testing for  & Rendani Dau, \\ & & reporting, and and & Elana Kuun, \\ & & menu items & Semaka Malapane, \\ & & controllers and updates &  Antonia Michael \\ & & & Isabel Nel \\   [1ex] 
 \hline
 \end{tabular}
\end{table} 

\pagebreak
%%---------------------------------  INTRODUCTION -------------------------------------------
\section{Introduction}
This document contains the   testing for the Resolve Cafeteria Management System that will be created for Software Engineering (COS 301) at the University of Pretoria 2015, by the group T-RISE. In this document we will thoroughly discuss and layout the project's testing to provide a clear view of the system as a whole. An agile approach is being followed which involves an interactive and incremential method of managing and designing the system. 

%% ------------------------------ VISION ------------------------------------------------------
\section{Vision}
The vision of this project is to implement a flexible, pluggable, fully functional software application that will be maintainable, with detailed supporting documentation and an instruction manual for the Cafeteria Management System. This system will assist in managing the cafeteria's inventory/stock, executing orders from the cafeteria, generating bills and sending these to the appropriate parties and facilitating payments for access cards (or the use of unique access card numbers). 

%%---------------------------------- BACKGROUND -----------------------------------------
\section{Background}
\subsection{The current situation/ problems the client currently experience}
As specified in the project proposal document from Resolve, the cafeteria is currently cash only and does not accept bank cards or electronic payments. This is it inconvenient for employees as they have to carry around cash if they want to purchase anything from the cafeteria. Employees may choose to go to an external food outlet where they can pay with their preferred method of payment, which uses time and fuel. Thus, this means the cafeteria does not achieve the maximum amount of income which hinders its growth and improvement.\\ A problem with the cafeteria itself is that certain meal items are hardly in stock due to either lack of ingredients to make the meal or under estimating the quantity of the meal item required.

\subsection{How the aforementioned problems will be alleviated by the CMS}
The Cafeteria Management System will provide a means to accept payments from employees, at the canteen, using their employee access cards or access card numbers, with an amount being deducted from their salary at the end of the month.  The option of cash payments ,however, will not be discarded. At the end of each month, the bill for the month will be sent to either payroll, to the employee, or to both. This option is thus configurable from the user's profile. The employee can also set a spending limit for each month. There will also be a system wide limit that users cannot exceed.
\\
The system will predict which inventory items needs to be bought for the next week in order to avoid the "out of stock" situation described above. The system will also enforce that the cafeteria manager 
\\
%%-------------TESTING--------------------------------------
\section{Test Plan}

\subsection{Introduction}
The scope of the testing: This document will be used be the Team T-RISE to elaborate and substantiate the unit tests conducted for the various use cases/ modules of the Resolve Cafeteria Management System. Unit tests using mocks and Integration tests using actual data from the system have been conducted and will be explained in the document below. Tests for the server side functionality as well as the client side functionality were executed.

\subsection{Technologies used for testing}
Testing was done during the construction of the various functions to ensure that the code is working at all times, and that only fully working functions are used in other functions going forward. PhantomJS was used to run the Mocha, Karma and Jasmine tests automatically, hence the tests for the different files could be run consecutively without being manually executed separately. The reason the testing frameworks were used was due to the simple set up that it required and due to its compatibility with PhantomJs. In addition the syntax of these was simple and intuitive.  Karma in particular is the official Angular Js test runner, hence it was used. Mocha, is based on node.js and hence was used for this reason. 
\\
\subsection{How to run the unit tests}
To execute the tests one must first ensure that the Mongo database is running. After this, one opens a separate terminal inside the Cafeteria Management System. Then sudo npm test is run which executes PhantomJs to run the tests automatically.  The output will indicate whether the tests are passing or failing as well as the different descriptions for each test. Server side and client side tests are executed.

\subsection{Testing approach - Client Side}
The unit tests written for the Authentication module were done to test the functions inside authentication.client.controller.js and the superuser.client.controller.js file.  The register/ signup functionality also resides in the authentication.client.controller.js file. The unit tests written for the Manage Profile module were done to test the settings.client.controller.js and the password.client.controller.js files, which contained the code to implement the manage profile module.
 
The functions tested for the Authentication module include signin() and signup(), located in the authentication.client.controller.js file, assignRoles(), setSystemWideLimit(), and setCanteenName() located in the  superuser.client.controller.js file.
 
The functions changeUserPassword() from the settings.client.controller.js and askForPasswordReset() from password.client.controller.js were tested. resetUserPassword() was not tested due to the fact that each time a user clicks Forgot Password, an email is sent to the user 
 
\subsection{Testing approach - Server Side}
The various functions from each module were tested on the server side, using the save() function to test if the items were able to be saved on the database. For the unit tests, mock objects were created, storing data under the different table column headings.
\\
The pre and post conditions, test cases and the result of the tests will now be discussed.

\subsection{Tests conducted/ Test coverage - Client Side}
Test for the controllers: 

\subsubsection{Profile Module - settings.client.controller.js}
The tests can be found in settings.client.controller.test.js
1.) Change Password
Pre conditions: Valid password entered
\\Post conditions: Password has changed
\\ Test Case 1: \$scope.changeUserPassword() should let user change their password if a valid one was entered: Pass
\\ Test Case 2: \$scope.changeUserPassword() should send an error message if new password is too short : Pass
\\ Test Case 3: \$scope.changeUserPassword()  should send an error message if the current password is incorrect: Pass
\\ Test Case 4: \$scope.changeUserPassword() should send an error message if the passwords do not match: Pass
 
\subsubsection{Authentication Module - authentication.client.controller.js }
The tests can be found in authentication.client.controller.test.js
1.) Sign in : \\
Pre conditions: Valid credentials have been entered, the user exists on the system
\\ Post conditions: User has been signed in
\\  Test Case 1:\$scope.signin() should login with a correct user and password: Pass
\\ Test Case 2: \$scope.signin() should fail to log user in if nothing has been entered: Pass
\\ Test Case 3: \$scope.signin()  should fail to log user in with wrong credentials: Pass

2.) Sign up: \\
Pre conditions: User entered valid credentials, Username/ Employee Id does not exist on the system
\\ Post conditions: User is signed up to use the system \\
Test Case 1 - \$scope.signup()  should register user with correct data: Pass \\
Test Case 2 - \$scope.signup()  should fail to register with duplicate Username: Pass

\subsubsection{Manage System Module - superuser.client.controller.js }
1.) Assign Roles - \\
Pre conditions: User id entered by superuser exists in the system \\
Post conditions: Role has been assigned to user by superuser \\
Test Case 1: \$scope.assignRoles() should let superuser assign roles because the user id entered exists on the system: Pass
\\ Test Case 2:  \$scope.assignRoles() should not let superuser assign roles because user id entered does not exist on the system: Pass \\
Test Case 3: \$scope.assignRoles() should redirect if superuser role is assigned: Pass \\
 
2.) Set system wide limit - \\
Pre conditions: Limit is valid \\
Post conditions: Limit has been set \\
Test Case 1: \$scope.setSystemWideLimit() should allow superuser to set limit: Pass

3.) Set canteen name-\\
Pre conditions: N/a \\
Post conditions: Canteen name set successfully \\
Test Case 1: \$scope.setCanteenName()  should let user set canteen name: Pass \\

4.) Change Employee ID- \\
Pre conditions: New employeeID doesn't exist currently, new employeeID field is not blank \\
Post conditions: Employee ID updated \\
Test Case 1: \$scope.changeEmployeeID()should not change employeeID if the user is not in the database: Pass \\
Test Case 2: \$scope.changeEmployeeID() should change employee ID: Pass \\

5.) Search Employees \\
Test Case 1: \$scope.searchEmployee() should search for employee by ID: Pass \\
Test Case 2:  \$scope.searchEmployee() should not search for employee if emp id blank: Pass \\
Test Case 3: \$scope.searchEmployee() should not search for employee by ID: Pass \\

\subsubsection{Manage Cafeteria Module - menuItems.client.controller.js }
1.)Update Menu Items \\
Pre conditions: information entered is valid\\
Post conditions: menu item updated\\
Test Case 1: should update menu item: Pass \\
Test Case 2: should not update menu item: Pass \\

2.) Search Menu items \\
Pre conditions: menu item exists in the database \\
Post conditions: menu item found \\
Test Case 1: \$scope.updateMenuItem() should update menu item: Pass \\
Test Case 2: \$scope.updateMenuItem()should not update menu item: Pass \\
Test Case 3: \$scope.searchMenu() should search and find menu item: Pass \\
Test Case 4: \$scope.searchMenu() should search and NOT find menu item: Pass \\

3.) Add Menu item \\
Pre conditions: menu item does not exist\\
Post conditions: menu item added\\
Test Case 1: \$scope.createMenuItem() should create menu item: Pass\\
Test Case 2: \$scope.createMenuItem() should fail to create menu item: Pass\\

4.) Load menu items \\
Pre conditions: menu items must exist in the database\\
Post conditions: menu items will be loaded \\
Test Case 1: \$scope.loadMenuItems() should load menu items: Pass \\ 

5.) Create a menu category \\ 
Pre conditions: Category does not exist \\
Post conditions: Created successfully, added to database\\
Test Case 1: \$scope.createMenuCatagory() should create category: Pass \\
Test Case 2: \$scope.createMenuCatagory() should fail to create category: Pass \\

6.) Load Menu Categories \\
Pre conditions: One/ more than one menu category exists in the database \\
Post conditions : Categories have been loaded and are displayed on navigation bar 
Test Case 1: \$scope.loadMenuCategories should load categories: Pass \\
Test Case 2: \$scope.loadMenuCategories should fail to load categories: Pass \\

7.)Check Stock \\
Pre conditions: There are menu items and inventory items saved\\
Post conditions: The inventory needed for the menu item is either enough to be used for the quantity of items specifed or not. If not it is marked as out of stock\\
Test Case 1: \$scope.checkStock should successfully perform a check: Pass \\  

8.) Delete Menu Items \\
Pre conditions : Menu item exists in the database \\
Post conditions: Menu item successfully deleted or error message returned \\
Test Case 1: \$scope.deleteMenuItem should delete item from menu \\
Test Case 2: \$scope.deleteMenuItem should fail to delete item from menu \\

\subsubsection{Place order Module - orders.client.controller.js, cashier.client.controller.js , finance.client.controller.js}
\begin{enumerate}

\item The first set of tests can be found in orders.client.controller.test.js.\\
Pre Conditions: A valid user has logged in. The user has sufficient funds to account for the order placed.\\
Post Conditions: The order is sent through to the cashier. \\
Test Case 1: \$scope.placeOrder() should not allow order to be placed: Pass\\
Test Case 2: \$scope.placeOrder() should allow order to be placed: Pass\\

\item The second set of tests can be found in cashier.client.controller.test.js.\\
Pre Conditions: The user facilitating the orders is the cashier and is hence on the process orders page that is only accessible to the cashier\\
Post Conditions: The order has been marked as ready or closed.The user is notified when order is ready\\
Test Case 1: \$scope.markAsReady() should let user know order is ready: Pass\\
Test Case 2: \$scope.markAsReady() should not let user know order is ready when incorrect parameters sent to function: Pass \\
Test Case 3: \$scope.markAsCollected() should  mark the order as collected successfully: Pass \\
Test Case 4: \$scope.markAsCollected() should not mark the order as collected successfully: Pass \\
Test Case 5: \$scope.markAsPaid() should  mark the order as paid successfully: Pass \\
Test Case 6: \$scope.markAsPaid() should  mark the order as collected successfully: Pass \\
Test Case 7: \$scope.getOrders() should get a list of orders where status is open: Pass \\
Test Case 8: \$scope.getOrders() should not get a list of orders where status is closed: Pass \\

\item The third set of tests can be found in finance.client.controller.test.js.\\
Pre Conditions: The user facilitating the orders is the finance manager and is hence on the process orders page that is only accessible to the finance manager. \\ The employee ID has been entered and is valid\\
Post Conditions: The bill corresponding to the Employee ID entered is generated\\
Test Case 1: \$scope.getUserOrders() should get users orders: Pass\\
Test Case 2: \$scope.getUserOrders() should not get users orders: Pass \\

\end{enumerate}

\subsection{Tests conducted/ Test coverage - Server Side}

\subsubsection{Authentication Module - user.server.model.js}
The tests for this file can be found in user.server.model.test.js.
Pre Conditions: The fields are all filled in with valid credentials, The system starts off with no users \\
Post Conditions: The user has been saved in the database /  Error message sent\\ 
Test Case 1: should begin with no users: Pass\\
Test Case 2: should be able to save without problems: Pass\\
Test Case 3: should fail to save an existing user again: Pass\\
Test Case 4: should be able to show an error when trying to save without first name: Pass\\
Test Case 5: should be able to show an error when trying to save without last name: Pass\\
Test Case 6: should be able to show an error when the password is too short: Pass\\
Test Case 7: should be able to show an error when the passwords do not match: Pass\\
Test Case 8: should show an error when the email does not contain an @ sign and the email is in correct format: Pass\\

 \subsubsection{Manage Inventory Module - inventory.server.model.js}
 The tests for this file can be found in inventory.server.model.test.js. \\
Pre Conditions: The inventory item does not exist currently in the database,
all the fields have been filled in with valid information, the system does not have inventory stored at the start \\
Post Conditions: The inventory item has been added to the database/  Error message sent\\
Test Case 1: should begin with no inventory : Pass\\
Test Case 2: should be able to save without problems : Pass\\
Test Case 3: should fail to save an existing inventory item again : Pass\\
Test Case 4: should be able to show an error when trying to save without inventory name : Pass\\
Test Case 5: should be able to show an error when trying to save without quantity : Pass\\
Test Case 6: should be able to show an error when trying to save with incorrect unit : Pass\\
 
  \subsubsection{Place Orders Module - menuItem.server.model.js}
 
The tests can be found in menuItem.server.model.test.js.\\
Pre Conditions: There are no items in the menu database to start off with, All required fields are filled in with valid information\\
Post Conditions: The menu item has been saved to the database/ Error message sent\\
Test Case 1: should begin with no menu items: Pass\\
Test Case 2: should be able to save without problems: Pass\\
Test Case 3: should fail to save an existing menu item again: Pass\\
Test Case 4: should be able to show an error when trying to save without menu item name: Pass\\
Test Case 5: should be able to show an error when trying to save without description: Pass\\
Test Case 6: should be able to show an error when trying to save without price: Pass\\
Test Case 7: should be able to show an error when trying to save without category: Pass\\
Test Case 8: should be able to show an error when trying to save with incorrect category: Pass\\
Test Case 9: should be able to show an error when trying to save without ingredients: Pass\\

\subsubsection{Database Storage Test - test.server.model.js}
The tests for this file can be found in test.server.model.test.js.
Pre Conditions: Name has been filled in\\
Post Conditions: Item saved to database/ Error message sent: Pass \\
Test Case 1: should be able to save without problems: Pass\\
Test Case 2:  should be able to show an error when try to save without name: Pass\\

\subsubsection{ Reporting module- users.finance.server.controller.js}
The tests for this file can be found in users.finance.server.controller.tests.js.
Pre Conditions: User has been found, user ID exits in the database \\
Post Conditions: Report has been downloaded in pdf format\\
Test Case 1: should render jsreport: Pass \\

\subsubsection{ Authentication module- test.server.routes.js}
The tests for this file can be found in test.server.routes.test.js.
Pre Conditions: User has logged on, User has signed in\\
Post Conditions: Test instance added/deleted/updated/ list of items retrieved or Error message sent \\
Test Case 1: should be able to save Test instance if logged in: Pass\\
Test Case 2: should not be able to save Test instance if not logged in: Pass\\
Test Case 3: should not be able to save Test instance if no name is provided: Pass\\
Test Case 4: should be able to update Test instance if signed in: Pass\\
Test Case 5: should be able to get a list of Tests if not signed in: Pass\\
Test Case 6: should be able to get a single Test if not signed in: Pass\\
Test Case 7: should be able to delete Test instance if signed in: Pass\\
Test Case 8: should not be able to delete Test instance if not signed in: Pass


\end{document}
