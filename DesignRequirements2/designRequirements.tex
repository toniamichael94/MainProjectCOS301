\documentclass[a4paper,12pt]{article}
\begin {document}
\section{Authentication}
The authentication controller allows existing users to log in and new users to sign up.
\subsection{authentication.client.controller}
 \begin{itemize}
 \item Sign up \\
 \$scope.signup = function()\\
 -Usage\\
 Method that allows a new user to sign up.
 \item Sign in\\
  \$scope.signin = function()\\
 -Usage\\
 Method that allows an existing user to sign in.
 \end{itemize}
 \subsection{users.authentication.server.controller}
 \begin{itemize}
 \item Sign up\\
  exports.signUp = function(req,res)\\
  -Usage\\
  This method ensures that all the fields neccessary to sign up were completed correctly. It returns an error message if one or more of the fields were entered incorrectly, otherwhise it adds the user deatils to the database.
  \item Sign in\\
exports.signin = function(req, res, next) \\
-Usage\\
This method allows an existing user to sign it if the user entered his/her details correctly. It performs a check to see if the user id and the passoword is correct. If the details are incorrect an error message is displayed.
\item Sign out\\
-Usage\\
This method lets a user sign out of the system.
  \item Check super user\\
  -Usage\\
  This method performs a check to confirm whether there is an existing super user or not. If there is no super user, it creates a super user and admin user, and sets the system wide limit to 5000. If these actions were not completed successfully the method displays an error message. This method also checks whether there is an existing admin user and creates one if there is not currently and admin user. An error message is displayed if this action was not performed successfully.
  
  \item add outh functions
 \end{itemize}
 
 \section{Settings}
 \subsection{settings.client.controller}
 The settings controller allows the user to view and edit profile information. The controller also perfomrs security checks to make sure that unauthorised users cannot access certain functions of the system. 
 \begin{itemize}
 \item Update user profile\\
 \$scope.updateUserProfile = function(boolean isValid)
 -Usage\\
 This method allows the user to update profile information. It displays an error if any of the information entered by the user is not in the correct format, or if the limit set by the user exceeds the system limit.
\item View user profile\\
 \$scope.viewUserProfile = function(boolean isValid)\\
-Usage\\

\item Change user password\\
 \$scope.chageUserPassword = function()\\
-Usage\\
This method lets the user change his/her password.

\item Get system limit\\
 \$scope.getSystemLimit = function()\\
-Usage\\
This method retrieves the system wide limit.

\item Search employee\\
 \$scope.searchEmployee = function(boolean isValid)\\
-Usage\\
This method allows the superuser to search for an employee, for example when a new role needs to be assigned to the employee.
This method is also used in finance where an employee is searched to display his/her bill.

\item Load employees\\
 \$scope.loadEmplyees = function()\\
-Usage\\

\item Check user role\\
 \$scope.checkuser = function()\\
-Usage\\
This method performs an authentication check to ensure that unauthorised users cannot access certain features. For example, a normal user will not be able to navigate to the manage cafeteria page.

\item Check if the user has a financial role\\
 \$scope.checkFinUser = function()\\
-Usage\\
The checkFinUser method performs a check to establish wheter the user has a finance role or not. It prevents unauthorised users from performing actions that can only be done by users with a finance role.
 \end{itemize}
 \section{Password}
 \subsection{password.client.controller}
 The password controller allows a user to reset his/her passoword if they forgot what their password is. 
 \begin{itemize}
\item Ask for password reset\\
  \$scope.askForPasswordReset = function()\\
 -Usage\\
 This method asks the user for his/her email and then sends the user a link to follow to reset his/her password. In the case where an email could not be sent an error message is displayed.
 \item Reset user password
  \$scope.resetUserPassword = function()\\
 -Usage\\
 This method allows the user to choose a new password.
 \end{itemize}
 \section{Inventory}
 \subsection{inventory.client.controller}
 The inventory controller handles all the main functionality concerning inventory, such as adding an anventory item.
 \begin{itemize}
 \item Add form field inventory\\
  \$scope.addFormFieldInventory = function()\\
  -Usage\\
  This method dynamically adds fields to the inventory page that are used to update the quantity of an inventory item. Each inventory item is displayed with the option to edit the quantity.
\item  Update inventory quantity\\
 \$scope.updateInventoryQuantity = function()\\
 -Usage\\
 This method updates the quantity of an inventory item.
 \item Delete inventory item\\
  \$scope.deleteInventoryItem = function()\\
  -Usage\\
  This method removes an inventory item from the database. It displays an error message when a user wants to remove an inventory item that is used by a menu item, in this case it displays an error message and does not remove the inventory item from the database.
  \item Load inventory items\\
   \$scope.loadInventoryItems = function()\\
   -Usage\\
   The load inventory items searches for and returns all the inventory items in the database. It displays an error message when it could not load all the inventory items.
   \item Create inventory item\\
    \$scope.create = function(boolean isValid)\\
    -Usage\\
    This method creates a new inventory item and stores it in the database. It displays an error message if all the neccessary fields are not entered correctly. 
    \item Search inventory\\
     \$scope.searchInventory = function(boolean isValid)\\
     -Usage\\
     This method searches for and returns a specific inventory item. It displays an error message if the inventory item could not be found.
     \item Update inventory\\
      \$scope.updateInventory = function(boolean isValid)\\
      -Usage\\
      This method updates a specified inventory item. It displays an error message if all the neccessary fields were not entered correctly or if the item could not be updated.
      \item Check if cafeteria manager\\
       \$scope.checkCMUser = function()\\
       -Usage\\
       This method performs a check to te confirm that the user has the cafeteria manager role. It prevents unauthorised users to perform actions that only the cafeteria manager is allowed to perform.
       
 \end{itemize}
 \end{document}