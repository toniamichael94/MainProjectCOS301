
%% Title Page 
\title{\Huge Testing Spec\\ 
	 Project: \\ 
	Cafeteria Management System: Reslove}
\author{
         \underline{T-RISE}\\
          Rendani Dau (13381467) \\
	Elana Kuun (12029522) \\
	Semaka Malapane (13081129) \\
	Antonia Michael (13014171) \\
	Isabel Nel (13070305)}

\date{\today}

\documentclass[12pt]{article}

\begin{document}
\maketitle
\break

%% Make table of contents
\tableofcontents
\break

%%now begin document

%%---------------------------------  INTRODUCTION -------------------------------------------
\section{Introduction}
This document contains the functional requirements specification, architecture requirements and testing for the Resolve Cafeteria Management System that will be created for Software Engineering (COS 301) at the University of Pretoria 2015, by the group T-RISE. In this document we will thoroughly discuss and layout the project's design requirements to provide a clear view of the system as a whole. An agile method is being followed so the following document focusses on the PlaceOrder and ManageProfile modules.

%% ------------------------------ VISION ------------------------------------------------------
\section{Vision}
The vision of this project is to implement a flexible, pluggable, fully functional software application that will be maintainable, with detailed supporting documentation and an instruction manual for the Cafeteria Management System. This system will assist in managing the cafeteria's inventory/stock, executing orders from the cafeteria, generating bills and sending these to the appropriate parties and facilitating payments for access cards (or the use of unique access card numbers). 

%%---------------------------------- BACKGROUND -----------------------------------------
\section{Background}
As specified in the project proposal document from Resolve - the cafeteria is currently cash only and does not accept bank cards or electronic payments. This makes it inconvenient for employees as they have to carry around cash if they want to purchase anything from the cafeteria. Hence, this is equivalent to purchasing from an external food outlet where they can also pay with their preferred method of payment. The employees have to hence use up fuel and time and lastly this does not bring in the maximum amount of income to the cafeteria, hindering its growth and improvement.\\

Resolve is therefore looking for a means to accept payments from employees for the canteen using their employee access cards or access card numbers, with an amount being deducted from their salary at the end of the month.\\

Resolve proposed the Cafeteria Management System to assist with this problem.
After our first meeting with the client, they brought to our attention that at times the cafeteria does not even have enough stock to provide some of the menu items, thus the managing of inventory or stock will also be part of the system. The system will also predict what inventory/stock needs to be bought for the next week in order to avoid such a problem. At the end of each month, the bill for the month will be sent to either payroll or to the employee. This option is configurable from the user's profile. The employee can also set a spending limit for each month for control purposes. The system will have its own maximum, such that users cannot set a limit that exeeds this. 


%%-------------DESIGN--------------------------------------
\section{Standards and conventions}

\subsection{Design standards}
 The diagrams are designed and created using UML. The main use case of the system is decomposed into components.

\section{Manage Profile}
\subsection{superuser.client.controller.js}
\begin{enumerate}
\item Declaration: \\ angular.module('users').controller('superuserController', ['$scope', '$http', '$location', '$window', 'Users', 'Authentication',
	function($scope, $http, $location, $window, Users, Authentication) { }
\item Methods: 
	\begin{itemize}
		\item Assign roles \\ \$scope.assignRoles = function(isValid) \{\} \\ \\
		Usage: Superuser can assign cashier, cafeteria manager, finance manager and admin roles
		\item Assign roles admin role \\ \$scope.assignRolesAdminRole = function(isValid) \{ \} \\ \\
		Usage: Admin user also has access to the assign roles functionality and serves as a back up superuser
		\item Change employee ID \\ \$scope.changeEmployeeID = function(isValid) \{\} \\ \\
		Usage: The superuser can change the employee ID of the users if the user signed up with the incorrect ID or if the company changes the IDs.
		\item Remove employee \\ \$scope.removeEmployee = function(isValid) \{\}
		\\ \\ Usage: The superuser is also able to remove users from the system, due to resignation or dismissal for example
		\item Search employee \\ \$scope.searchEmployee = function(isValid) \{ \}
		\\ \\ Usage: This function is used to retrieve employees from the system to be able to change their employe ID or remove them from the system
		\item Search employee ID \\ \$scope.searchEmployeeID = function(row) \{\}
		\\ \\ Usage: This function is used to retrieve employee IDs 
		\item Set system wide limit \\ \$scope.setSystemWideLimit = function(isValid)\{ \} \\ \\ Usage: This is used by the superuser to set the maximum monthly spending limit for all the users of the system.
		 \item Set canteen name \\ \$scope.setCanteenName = function(isValid)\{ \}
		\\ \\ Usage: The canteen name is configurable from the superuser's branding settings page.
		\item Check user \\ \$scope.checkUser = function()\{\}
		\\ \\Usage: This is a security function that checks to make sure that an authorized superuser is accessing the page and if this is not the case, the user will be redirected to the home page
		\item Load employees \\ \$scope.loadEmployees = function()\{\}
		\\ \\Usage: This function is used to dynamically populate the drop down menu for the change employee ID functionality
	\end{itemize}
\end{enumerate}

\section{Manage System}
\section{Manage Cafeteria}
\section{Manage Inventory}
\section{Place Order}
\section{Authentication}


  
\end{document}
