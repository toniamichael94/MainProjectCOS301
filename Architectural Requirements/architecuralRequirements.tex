\documentclass[a4paper,12pt]{article}
\usepackage[hidelinks]{hyperref}
\usepackage{graphicx}
\usepackage{float}
\usepackage{caption}
\usepackage{array}
\usepackage{tabu}

%% Title Page 
\title{\Huge Architectural Requirements Document Specification \\ 
	 Project: \\ 
	Cafeteria Management System: Resolve}
\author{
         \underline{T-RISE}\\
          Rendani Dau (13381467) \\
	Elana Kuun (12029522) \\
	Semaka Malapane (13081129) \\
	Antonia Michael (13014171) \\
	Isabel Nel (13070305)}





\date{\today}

%\documentclass[12pt]{article}

\begin{document}
\maketitle
\break

%% Make table of contents
\tableofcontents
\break

 \begin{tabu} to 0.8\textwidth { | X[l] | X[l] | }
 \hline
 \textbf{Document Title} & Architectural Specification Document \\
 \hline
 \textbf{Document Identification}  & Document 0.0.2 \\
 \hline
 \textbf{Author}  & Rendani Dau, Isabel Nel, Elana Kuun, Semaka Malapane, Antonia Michael \\
 \hline
 \textbf{Version} & 0.0.2\\
 \hline
 \textbf{Document Status} & Second Version - contains edited architectural requirements  \\
 \hline
 \end{tabu}

\begin{table}[h!]
\centering
 \begin{tabular}{||c c c c||} 
 \hline
 \textbf{Version} & \textbf{Date} & \textbf{Summary} & \textbf{Authors} \\ [0.5ex] 
 \hline\hline
 0.0.1 & 29 May 2015 & First draft contains architectural requirements & Rendani Dau, \\ & & & Isabel Nel, \\ & & & Elana Kuun, \\ & & & Semaka Malapane, \\ & & & Antonia Michael \\   [1ex] 
 \hline\hline
 0.0.2 & 30 July 2015 & Second draft contains edited architectural requirements & Rendani Dau, \\ & & & Isabel Nel, \\ & & & Elana Kuun, \\ & & & Semaka Malapane, \\ & & & Antonia Michael \\   [1ex] 
 \hline
 \end{tabular}
\end{table}

\pagebreak

%%now begin document

%%---------------------------------  INTRODUCTION -------------------------------------------
\section{Introduction}
This document contains the architectural requirements for the Resolve Cafeteria Management System that will be created for Software Engineering (COS 301) at the University of Pretoria 2015, by the group T-RISE. In this document we will thoroughly discuss and layout the project's architectural requirements to provide a clear view of the system as a whole.  

%% ------------------------------ VISION ------------------------------------------------------
\section{Vision}
The vision of this project is to implement a flexible, pluggable, fully functional software application that will be maintainable, with detailed supporting documentation and an instruction manual for the Cafeteria Management System. This system will assist in managing the cafeteria's inventory/stock, executing orders from the cafeteria, generating bills and sending these to the appropriate parties and facilitating payments for access cards (or the use of unique access card numbers). 

%%---------------------------------- BACKGROUND -----------------------------------------
\section{Background}
As specified in the project proposal document from Resolve - the cafeteria is currently cash only and does not accept bank cards or electronic payments. This makes it inconvenient for employees as they have to carry around cash if they want to purchase anything from the cafeteria. Hence, this is equivalent to purchasing from an external food outlet where they can also pay with their preferred method of payment. The employees have to hence use up fuel and time and lastly this does not bring in the maximum amount of income to the cafeteria, hindering its growth and improvement.\\

Resolve is therefore looking for a means to accept payments from employees for the canteen using their employee access cards or access card numbers, with an amount being deducted from their salary at the end of the month.\\

Resolve proposed the Cafeteria Management System to assist with this problem.
After our first meeting with the client, they brought to our attention that at times the cafeteria does not even have enough stock to provide some of the menu items, thus the managing of inventory or stock will also be part of the system. The system will also predict what inventory/stock needs to be bought for the next week in order to avoid such a problem. At the end of each month, the bill for the month will be sent to either payroll or to the employee. This option is configurable from the user's profile. The employee can also set a spending limit for each month for control purposes. The system will have its own maximum, such that users cannot set a limit that exeeds this. 

\section{Architecture Requirementss}
The software architecture requirements include the access and integration requirements, quality
requirements, and architectural constraints. These points will be thouroughly discussed below.

%%---------------------------------- ARCHITECTURAL RESPONSIBILITIES----------------------------
\section{Architectural Responsibilities}
\begin{itemize}
\item The system must allow a user to view the menu without being logged in
\item The system must allow a user to select items from the menu and proceed to place the order
\item The system should not process an order unless a user is logged in
\item On placement of an order the system should send the order details to the cafeteria
\item The system should notify the user when their order is ready
\item The system should send a bill to payroll or the user  once a month or both (depending on preferance)
\item The system should allow a user to manage their profile by editing their password, email and spending limit
\item The system should allow configeration and branding by a super user or an  admin user
\item The system should allow management of inventory and management of the cafeteria
\item The system should provide casheir functionality
\item The system should provide for multiple deployment
\item The system must be usable, scalable, reliable and auditable
\end{itemize}

%% ---------------------------------  QUALITY REQUIREMENTS -------------------------------------
\section{Quality Requirements}

%%--------------SECURITY------------------
\subsection{Security}
\subsubsection{Reasons for security in quality requirements}
 \begin{itemize}
 \item This is the most important requirement for the Cafeteria Management System because the system involves storing confidential details and deducting the bill from the users salary. The user must hence have full control over the amount spent.  
\item There are also superuser and admin roles who have certain functionality assigned to them, dealing with the confidential details of the users and the cafeteria and hence ordinary users must not be able to use the route to the superuser to access the superuser functionality. Hence, security must be in place.
 \end{itemize}
 \subsubsection{Strategies to achieve this quality requirement}
 \begin{itemize}
 \item Detecting attacks by determining message integrity, event logging and analysis, scanning for attack signatures and auditing sensitive events (Solms, 2014).
\item Resisting attacks by limiting access and exposure. This can be done by minimizing access channels, minimizing access domains, authentication, confidentiality, assuming external resources as untrusted, minimizing hosting of external resources, additonal security layers for valuables.
\item Recovering from attacks by dropping connections and requests. In addition, updating access rules and restoring states.
 \end{itemize}
\subsubsection{Patterns to achieve these strategies}
 \begin{itemize} 
 \item Layering
\end{itemize} 
Layering can enforce security due to seperation into layers, in which different users have different levels of access to the layers, depending on their roles in the system
Layering can be used within the subsections of MVC, i.e. the Control and Model layer can be layered to further divide concerns and restrict different users' access on those layers, providing tighter security.


\subsection{Usability}
 \subsubsection{Reasons for quality requirements}
 \begin{itemize}
 \item Since the users of the cafeteria system will be in different divisions within the company and have different technological skill levels, it should not be hard for new users to become familiarized with the system
 \item Due to the fact that Resolve has not had such a system in place before, the novice user would not have had exposure to such a system therefore the system should also be rememberable hence it must also be understandable. 
	\item It should take at most 30 minutes for an average user to be familiar with the system
 \end{itemize}
 \subsubsection{Strategies to achieve this quality requirement}
 \begin{itemize}
 \item Various goals of usability requirements are firstly, that the that the interface is intuitive, i.e. easy to navigate and understand, that the buttons and icons are self explanatory for the primary users.
 \item The interface must also not be a cluttered, frustrating and overwhelming one. 
 \item Ease of learning is also an important goal here such that users who have never used such a system can catch on easily and such that users who regularly use other discussion boards will not get confused and displaced. 
 \item The system must also be task efficient, i.e. if users access this space regularly, long tedious processes and other admin must be avoided.
\item Also, the colour schemes, functionality and interactiveness of the interface and system must contribute to this task efficiency. 
\item Different usability tests can be conducted such as handing out paper prototypes of different interface designs, and questionnaires getting feedback from the sample of people that were consulted in the survey. Problems with the different interfaces can be picked up during the usability testing phase, as indicated by the sample of users consulted, such that the final product will be much more user friendly. (http://www.usability.gov/what-and-why/usability-evaluation.html)
 \end{itemize}
 \subsubsection{Patterns to achieve these strategies}
 \begin{itemize}
 \item MVC
 \item Layering
\end{itemize}
MVC is a suitable pattern because the user will only need to interact with the front end interface, rather than dealing with the technical aspects of the back-end system. Another reason for this is that the developers allocated to working on the View will have the sole focus of making it usable.
Layering can be used within the subsections of MVC, i.e. the Control and Model layer can be layered to further divide concerns and allow different people to work on those layers.

\subsection{Reliability}
\subsubsection{Reasons for quality requirement}
	\begin{itemize}
	\item The reason we have placed great importance on reliability is because a large scale of users (Resolve employees) will be using the system concurrently in the one hour lunch break thus the system should be able to notify each user separately when their orders are ready and a bill should be generated for each user
	\item The system should ensure that the correct amount is deducted from the user's account and that the user's set limits are adhered to.
	\item Reliability plays a key role, ensuring that all functions work as the user expects them, when the user requires to use the system.
	\item It must hence have a maximum of 2 or 3 hours down time a week.
	\item The system should be reliable in terms of ensuring that the different functionality is assigned to users with different roles.  
	\end{itemize}
	\subsubsection{Strategies to achieve this quality requirement}
	\begin{itemize}
	\item Firstly, the prevention of faults. This is done by testing the system thoroughly, using resource locking as well as removing single points of failure. (Solms, 2014)
 \item Secondly, detection of faults, which is achieved through deadlock detection, logging, checkpoint evaluation and error communication to name but a few. Recovering from faults also falls under reliability. This is done by passive redundancy, maintaining backups and checkpoint rollbacks. (Solms, 2014)
 \end{itemize}
 \subsubsection{Patterns to achieve these strategies}
 \begin{itemize}
 \item MVC
 \end{itemize}
 The MVC pattern can  be used for reliability, because since the different layers are clearly separated, hence particular teams are focused on working on each layer, making the system more reliable. 
 \subsection{Auditability}
 \subsubsection{Reasons for quality requirements}
 \begin{itemize}
 \item Any action performed on the system should be traceable back to the person who made these changes and when these changes were made.
 \item In the event of a system crash, it should be possible to roll the system back to a previous working state
 \item The super user should be able to view every other user's activities. This is a part of the monitorability aspect of the system.
 \end{itemize}
 \subsubsection{Strategies to achieve this quality requirement}
 \begin{itemize}
 \item System should have log files running at all times to track all transactions made by users. 
 \item Time stamps should be added to document time and date information of the activities done so that the system can trace through the
information when needed, such as the events that precede a system crash or unauthorized access
that alters the system in any way.
\item System backup should allow rollback when needed.
\item ACID test can be carried out. Acid is an acronym that describes the properties of a database or system. The properties are:
	\begin{itemize}
	\item \textbf{Atomicity:} Defined as all or none situation referring to the processes that take place on the 
	   system. If something where to go wrong with a process such as posting on the system,
	   then the entire process has to be repeated or not at all.
	\item \textbf{Consistency:} All processes must be completed. No process can be left in a half-finished state,
	     if a failure is detected in a process then the entire process has to be rolled back.
	\item \textbf{Isolation:} Keeps process/transactions separate from one another until they are finished.
	\item \textbf{Durability:} The system must keep a backup of its current state so as to roll back to it if
	    the system where to experience a system failure, crash or corruption of data due
	    to a security breach.
	\end{itemize}
 \end{itemize}
 \subsubsection{Patterns to achieve these strategies}
 \begin{itemize}
 \item MVC
 \item Layering
\end{itemize} 
 MVC is a suitable pattern because it provides auditability through logging all filter inputs and outputs (off queues). Layering is a suitable pattern because each separate layer can be audited and monitored individually, rather than auditing the system as a whole.
  
%\item Scalability
\subsection{Scalability}
	\subsubsection{Reasons for quality requirement}
	\begin{itemize}
	\item This is an important requirement due to the fact that a large volume of Resolve employees will be using the system possibly during lunch hour break and hence the system needs to support all these users concurrently. 
	\item In saying this, the system must allow each user to order multiple items and process orders per user concurrently.
	\item With this, we can assume that there will be in excess of 500 users meaning that the system has to have the ability to handle at least 500 concurrent users at peak times.
	\end{itemize}
	\subsubsection{Strategies to achieve this quality requirement}
	\begin{itemize}
		\item We will need to firstly ensure that existing resources are managed efficiently, i.e. reducing the load using efficient storage, processing,  and persistence. In addition, we will need to ensure that the load is spread across resources and time, using methods of load balancing to spread load across resources as well as using scheduling and queueing to spread load across time.
		\item Secondly, the resources can be scaled up by increasing storage, increasing processing power and increasing the capacity of communication channels.
		\item Lastly, resources can be scaled out by means of using external resources, using commoditized resources and distributing tasks across specialized resources.
		\end{itemize}
	\subsubsection{Patterns to achieve these strategies}
	 \begin{itemize}
		\item Concurrency Master-Slave 
	\end{itemize}
We chose this pattern here due to the concurrency of the system, meaning that a large number of users must be able to access the system at a time.
%\end{enumerate}
\subsection{Nice-To-Have}
	\begin{enumerate}
		\item Performance
		\item Integrability
		\item Flexibility
		\item Maintainability
\end{enumerate}

%% --------------------------------- ARCHITECTURE REQUIREMENTS ---------------------------------
\section{Integration Requirements}
%% ---------- ACCESS CHANNEL REQUIREMENTS ------------
\subsection{Access Channel Requirements}
In this section we will discuss the requirements for the different channels through which the system can be accessed by firstly, people (users - client side) and systems (server-side).


%%--------HUMAN ACCESS CHANNELS------
\subsubsection{Human Access Channels}
The Cafeteria Management System will be accessed by the different users via the online web page (web interface) or through the mobile application (if one is developed). The web interface will be accessible through all the standard web browsers such as Mozilla Firefox, Google Chrome and Microsoft Internet Explorer. The mobile application will be accessible on multiple platforms including the standard IOS/Anroid platforms. Different services will be available to different users. There are five types of users: Super User, Cafeteria Manager, Casher, Normal User, and Resolve Admin. These will be discussed below. \\

\paragraph{Super User\\}
The super user will be the only administrative user that will have global access to all the functionality of the Cafeteria Management System, in particular the Super User will have access to the branding of the Cafeteria Management system (changing the logo and so forth) . The super user will hence have access to all the functionality of all the other users listed below.

\paragraph{ Cafeteria Manager\\}
The cafeteria manager will have the ability to view his/her own profile, edit his/her profile, and place orders. This user will also be able to add and edit menu items, view the orders placed by the users of the system, view the inventory, and add or remove inventory. 

\paragraph{ Cashier\\}
The cashier will be able to view his/her profile, edit his/her profile, view the orders placed, and mark off finished orders that are finished and have been collected. The cashier will also be able to make purchases, check inventory and add or remove inventory. Removal of inventory will be done in situations where stock has expired or depleted.

\paragraph{ Normal User\\}
The normal user will be a Resolve employee registered on the Cafeteria Management System.  A normal user will only be able to view his/her profile, edit his/her profile, place orders, check if their order is ready, and view/print their balance reports and account history.

\paragraph{ Resolve Admin\\}
The resolve admin user will be able to view all the registered users, their account history and their outsanding balances. This is for administrative and financial purposes . This role has been requested by the Resolve team.

%%--------SYSTEM ACCESS CHANNELS -------
\subsubsection{System Access Channels}
The different technologies selected will be used to support the access channels effectively. We will be using NodeJs running on an Express server and the server needs to be connected to the Mongo database on which various data will be stored and retrieved. This data will be transfered from the server to the respective node modules and so forth. 
The integration channels will also be accessable by the mobile applications, such as Phone Gap, which is the program we will be using to help us convert our web interface into a mobile application.
\begin{itemize}

\item The system will have to integrate with the Mongo database, retrieving information of the employees - such as contact information to notify the user that an order is ready, get inventory or stock and so forth. 

\item The system will also have to integrate with the server to pass information to and from the database.
\end{itemize}

\subsection{Protocols}
\subsubsection{HTTP - Hypertext Transfer Protocol}
Integration with this protocol will occur at a high level and typically be handled by libraries or browser-clients etc.
\textbf{To be used for:	}
	\begin{itemize}
	\item{All data transferred between users and the server on which the system is hosted.}
	\item{Transfer of miscellaneous data such as HTTP error codes to ensure both servers and clients are aware of the state of data transfers and its results}
	\end{itemize}
	
\subsubsection{TCP - Transmission Control Protocol}
 For establishing network connections between the user computers and the system server. Streams of data can then be exchanged between the connected hosts. Error detection, faulty transmission of data, resending of data etc. will all be done using TCP (Davids). Integration with this protocol will occur at a high level and will typically be handled by libraries or operating system functions.
 
\subsubsection{SMTP - Simple Mail Transfer Protocol}
This protocol will be used to handle e-mail communication, specifically notifications to users when their orders are ready, as well as the sending of bills to both users and payroll. It addresses Security as a quality requirement since it incorporates SMTP-Authentication defined by RFC 2554 (Meyers, 1999) which enhances the security of the protocol.

%% ----------ARCHITECTURE CONSTRAINTS----------------
\subsection{Architecture Constraints}
\subsubsection{Technologies}
Technologies we will be using in the creation of the Cafeteria Management System includes the following: 

\begin{itemize}
  \item \textbf{HTML}: The Software system will be mainly web-based.

  \item \textbf{JavaScript together with AngularJS  and NodeJS}: this will enable us to add extra functionality to our web page and modularise the system thus also helping us to implement dependency injection. For creating reports we can also import ReportingJS to create visually pleasing reports that is logically structured

 \item \textbf{CSS together with BootStrap}: which will allow us to style our page and also make it interactive.

 \item \textbf{Mongo DB}: MongoDB for our database which goes extremely well with NodeJS. 

 \item \textbf{Express server}: will be set up as our server that will host the system.

 \item \textbf{Phone gap}: Phone gap will be used to convert our web page into a usable application which will then look like the online webpage that will run like a web interface in the background but  will seem like a mobile application to the user that will be accessible from multiple platforms.

\item \textbf{Git and GitHub}: A version control software and a repository website that will be used to host the source code of the Cafeteria Management System. Reasons for the use of Git is its ease of use and because it is free and open source. (http://git-scm.com/).

\end{itemize}
The above mentioned technologies will be our basis we will create our system on, but as we are busy building the Cafeteria Management System we will add other technologies as needed. 

\subsubsection{Operating Systems}
Since server side of the Cafeteria Management System will be an Express server which is based on JavaScript, the CMS server will be able to run on both Windows and Unix platforms, given that they both support JavaScript and NodeJS.\\
\\
The client side of the Cafeteria Management System will be able to run on almost all operating systems that can run a modern web browser such as Mozilla Firefox, Google Chrome or Internet Explorer (for example, Microsoft Windows, Apple Mac OSX or any distribution of Linux or BSD).

\section{Architectural Patterns}
		For the design of The Cafeteria Management System, two patterns are considered: the \textbf{MVC (Model-View-Controller) pattern} and the \textbf{Layering architectural pattern}. For concurrency, the \textbf{Master-Slave pattern} is used.\\
		The top layers of the Model and the Controller are interfaces for the MVC architecture and the layers below provide the functionality.\\
		MVC is considered because:
		\begin{itemize}
			\item It provides modularity (i.e. the system's concerns are separated, thus easier to implement). (Solms, 2014)
			\item It allows for better maintainability (one can maintain the Model, View and Controller separately). (Solms, 2014)
			\item Testability (it is easier to test because of separated concerns, so the source of any problems are easy to identify). (Solms, 2014)
			\item Reuse (it is possible to take any component and reuse it where necessary). (Solms, 2014)
		\end{itemize} 
		Layering allows the Cafeteria Management System to have pluggable layers, which will allow the developers to replace layers as needed.
		This pattern allows for: 
		\begin{itemize}
			\item Improved cohesion. (Solms, 2014)
			\item Reduced complexity of the system. (Solms, 2014)
			\item Improved testability (which will allow for easier debugging). (Solms, 2014)
			\item Improved reuse and maintainability of the source code, because all the layers are individual and separate from one another. (Solms, 2014)
		\end{itemize}
		However, it should be mentioned that Layering has a performance overhead associated with it, as well as higher maintenance costs associated with the lower layers, because they impact the higher levels. (Solms, 2014). Given the benefits, however, the authors feel that the reduced performance and maintenance costs are a good compromise for reduced complexity and testability.\\
		\\
		For concurrency, the Master-Slave architectural pattern (Solms, 2014) is considered, because the system needs to accommodate a large number of users at a time.

\newpage
\section{Bibliography}
		\begin{itemize}
			\item \textit{Basic MVC Architecture}. [Online]. Available: <http://www.tutorialspoint.com/struts\_2/basic\\\_mvc\_architecture.htm> [Accessed 5 March 2015].
			
			\item Beal, V. n.d. \textit{Linux Server}. [Online]. Available: <http://www.webopedia.com/TERM/L/linux\_\\server.html> [Accessed 10 March 2015].

			\item Bien, A. 2009. \textit{9 Reasons Why Java EE 6 Will Save Real Money - Or How To Convince Your Management}. [Online]. Available: <http://www.adam-bien.com/roller/abien/entry/\\8\_reasons\_why\_java\_ee> [Accessed 10 March 2015].

			\item Borges, B. 2013. \textit{Reasons to why I'm reconsidering JSF}. [Online]. Available: <http://blog.brunoborges.com.br/2013/01/reasons-to-why-im-reconsidering-jsf.html> [Accessed 10 March 2015].
			
			\item Davids, N.\textit{The Limitations of the Ethernet CRC and TCP/IP checksums for error detection} [Online]. Available: 
			<http://noahdavids.org/self-published/CRC-and-checksum.html>[Accessed 8 March 2015].

			\item \textit{Git --loval-branding-on-the-cheap}. [Online]. Available: <http://git-scm.com/> [Accessed 10 March 2015].

			\item \textit{JavaServer Faces Technology}. [Online]. Available: <http://www.oracle.com/technetwork/java/javaee/javaserverfaces-139869.html> [Accessed 10 March 2015].

			\item Kabanov, J. 2011. \textit{Ed Burns on Why JSF is the Most Popular Framework}. [Online]. Available: <http://zeroturnaround.com/rebellabs/ed-burns-on-why-jsf-is-the-most-popular-framework/> [Accessed 10 March 2015].
			\item Meyers, J. 1999. \textit{SMTP Service Extension for Authentication} [Online]. Available: <http://tools.ietf.org/html/rfc2554> [Accessed 9 March 2015]

			\item \textit{Securing FTP using SSH} [Online]. Available: <http://nurdletech.com/linux-notes/ftp/ssh.html> [Accessed 9 March 2015]

			\item \textit{PostgreSQL: About}. [Online]. Available: <http://www.postgresql.org/about/> [Accessed 10 March 2015].
			
			\item \textit{Software Engineering}. [Online]. Available: <http://sesolution.blogspot.com/p/software-engineering-layered-technology.html>[Accessed 5 March 2015].
								
			\item Solms, F. 2014. \textit{Software Architecture Desgin.} [Online]. University of Pretoria: Pretoria. Available: <http://www.cs.up.ac.za/modules/courses/download.php?id=8565> [Accessed 9 March 2015].
			
			\item Solms, F. 2015. \textit{Java Persistence API (JPA).} [Online]. University of Pretoria: Pretoria. Available: <http://www.cs.up.ac.za/modules/courses/download.php?id=8640> [Accessed 7 March 2015].
			
			\item \textit{SQL Server 2008 R2 Express Database Size Limit Increased to 10GB}. [Online]. Available: <http://blogs.msdn.com/b/sqlexpress/archive/2010/04/21/database-size-limit-increased-to-10gb-in-sql-server-2008-r2-express.aspx> [Accessed 9 March 2015].
			
			\item \textit{Usability Evaluation Basics}. [Online]. Available: <http://www.usability.gov/what-and-why/usability-evaluation.html> [Accessed 5 March 2015].

			\item Wähner, K. n.d. \textit{Why I will use Java EE (JEE, and not J2EE) instead of Spring in new Enterprise Java Projects in 2012}. [Online]. Available: <www.kai-waehner.de/blog/2011/11/21/why-i-will-use-java-ee-jee-and-not-j2ee-instead-of-spring-in-new-enterprise-java-projects-in-2012/> [Accessed 10 March 2015].		

			\item \textit{Welcome to Apache Maven}. [Online]. Available: <http://maven.apache.org/> [Accessed 10 March 2015].
			
			

			

		\end{itemize}
\end{document}
