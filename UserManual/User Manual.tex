\documentclass[a4paper,12pt]{report}
\usepackage[hidelinks]{hyperref}
\usepackage{graphicx}
\usepackage{float}
\usepackage{caption}
\usepackage{array}
\usepackage{tabu}
\usepackage[tt]{titlepic}
\renewcommand\thesection{\arabic{section}}
%% Title Page 
\titlepic{ \centering\includegraphics[width=0.7\textwidth]{../images/dddd.png} }
%\documentclass[12pt]{article}
\title{\Huge User Manual \\ 
	Cafeteria Management System: Resolve}
\author{
         \underline{T-RISE}\\
          Rendani Dau (13381467) \\
	Elana Kuun (12029522) \\
	Semaka Malapane (13081129) \\
	Antonia Michael (13014171) \\
	Isabel Nel (13070305) \\ \\
	https://github.com/toniamichael94/MainProjectCOS301}

\date{\today}

\begin{document}
\maketitle
\break

%% Make table of contents
\tableofcontents
\break


 \begin{tabu} to 0.8\textwidth { | X[l] | X[l] | }
 \hline
 \textbf{Document Title} & User Manual \\
 \hline
 \textbf{Document Identification}  & Document 0.0.5  \\
 \hline
 \textbf{Author}  & Rendani Dau, Isabel Nel, Elana Kuun, Semaka Malapane, Antonia Michael \\
 \hline
 \textbf{Version} & 0.0.5 \\
 \hline
 \textbf{Document Status} & Fifth Version - contains view order page, place order and cashier page section\\
 \hline
 \end{tabu}

\begin{table}[h!]
\centering
 \begin{tabular}{||c c c c||} 
 \hline
 \textbf{Version} & \textbf{Date} & \textbf{Summary} & \textbf{Authors} \\ [0.5ex] 
 \hline\hline
 0.0.1 & 9 July 2015 &  First draft contains how to run system  & Rendani Dau, \\ & & & Elana Kuun, \\ & & & Semaka Malapane, \\ & & & Antonia Michael \\ & & & Isabel Nel, \\ & & & \\
 \hline 
 & & & \\
 0.0.2 & 20 July 2015 &  Second draft adding page  & Rendani Dau, \\ & & assistance and explanation & Elana Kuun, \\ & & of how to use & Semaka Malapane, \\ & & functionality on page &  Antonia Michael \\ & & & Isabel Nel \\   [1ex]  
 \hline 
 & & & \\
 0.0.3& 23 July 2015 &  Third draft containing  & Rendani Dau, \\ & & screenshots of the & Elana Kuun, \\ & & of different page & Semaka Malapane, \\ & & and their functionality &  Antonia Michael \\ & & & Isabel Nel \\   [1ex]  
 \hline
 & & & \\
 0.0.4& 3 August 2015 &  Fourth draft Added  & Rendani Dau, \\ & & Troubleshooting & Elana Kuun, \\ & & Section & Semaka Malapane, \\ & &  &  Antonia Michael \\ & & & Isabel Nel \\   [1ex]  
  \hline 
 & & & \\
 0.0.5& 28 August 2015 &  Fifth draft containing  & Rendani Dau, \\ & & screenshots of the & Elana Kuun, \\ & & of different page & Semaka Malapane, \\ & & and their functionality &  Antonia Michael \\ & & & Isabel Nel \\   [1ex]  
\hline
 \end{tabular}
\end{table}

\pagebreak
\pagebreak
%%now begin document
%%---------------------------------  INTRODUCTION -------------------------------------------
\section{Introduction}
This document contains the user manual for the Resolve Cafeteria Management System that will be created for Software Engineering (COS 301) at the University of Pretoria 2015, by the group T-RISE. 

%% ------------------------------ VISION ------------------------------------------------------
\section{Vision}
The vision of this project is to implement a fully functional software application that will be maintainable, with detailed supporting documentation and an instruction manual for the Cafeteria Management System. This system will, amongst others, assist in executing orders from the cafeteria, managing the cafeteria's inventory, generating bills, and perform various reporting tasks. 

%%---------------------------------- INTRO -----------------------------------------
\section{Background}
As specified in the project proposal document from Resolve, the cafeteria is currently cash only and does not accept bank cards or electronic payments. This makes it inconvenient for employees as they have to carry around cash if they want to purchase anything from the cafeteria. Employees may choose to go to an external food outlet where they can pay with their preferred method of payment, which uses time and fuel. Lastly, this means the cafeteria does not achieve the maximum amount of income which hinders its growth and improvement.\\

Resolve is therefore looking for a means to accept payments from employees  using their employee access cards or access card numbers, with an amount being deducted from their salary at the end of the month.\\

Resolve proposed the Cafeteria Management System to assist with this problem.
After our first meeting with the client, they brought to our attention that at times the cafeteria does not have enough stock to provide some of the menu items, thus managing the inventory will also be part of the system. The system will predict which inventory items needs to be bought for the next week in order to avoid such a problem. At the end of each month, the bill for the month will be sent to either payroll or to the employee, or both. This option is configurable from the user's profile. The employee can also set a spending limit for each month. There will also be a system wide limit that users cannot exceed.

%%---------------------------------- SYSTEM OVERVIEW -----------------------------------------
\section{System Overview}
The Cafeteria Management System is a system designed to assist users to order food efficiently from their office's canteen and to be notified when their order is ready for collection. The system will also assist cafeteria staff with keeping track of the orders in real time as well as managing inventory. The system will also provide for configuring the branding settings of the cafeteria. The system is intended to be used in a corporate environment whereby users have the option to allow their cafeteria expenses to be deducted from their salary or immediately pay for orders. In addition, the system will allow management to view the bill reports of the different users. All users will also be able to access their spending history, set favourites and other similar functionality which will all be explained in this user manual. \\

%%--------------------------------------SYSTEM CONFIGURATION ----------------------------------------
\section{System Configuration}
The system requires a Windows/Unix based host to run the server. This host must have NodeJS, MongoDB, Express server, Bower and grunt  installed (the installation of these technologies will be discussed later in the document). In addition, the host must be connected to the internet in order to allow any required dependencies to be installed and set up for the operating system environment. The configuration of the server requires an active email account to facilitate communication between the system and end users. \\ 
\\
End users will only require a PC equipped with a web browser such as Mozilla, Chrome or Internet Explorer, as well as an active internet connection.

%%--------------------------------------INSTALLATION  ----------------------------------------
\section{Installation}

%%------------------------PREREQUISITES ----------------------------
\subsection{Prerequisites}
For the programmer, who will maintain the code: \\
The Cafeteria Management System requires NodeJS and MongoDB to run. These are free and open source software and can be obtained from the following sites:\\
\url{https://nodejs.org/download/} \\
\url{https://www.mongodb.org/downloads} \\
The applications are available for both Windows and Unix environments and include setup guides on their respective web pages.\\

Once installed, NodeJS includes a package manager called NPM. This package manager will be available from the terminal and will be used to install all dependencies. The following dependencies have to be installed first (Run these commands one by one in the command prompt or terminal):
\begin{verbatim}
$npm install -g bower
$npm install -g grunt-cli
\end{verbatim}

After these commands have successfully installed the respective applications you can download the Cafeteria Management Software from the GitHub repository :
 \url{https://github.com/toniamichael94/MainProjectCOS301}
\\ \\
This can be done by cloning the repository onto a remote location on your PC, if you do not know how to clone a GitHub repository, please visit:\\
  \url{https://git-scm.com/book/en/v2/Git-Basics-Getting-a-Git-Repository}  under the section "How to clone an existing repository" you will find the GitHub documentation on how to do this.
\\
Once you have cloned the GitHub repository and installed the above mentioned technologies, please move on to the next section , which will take you step by step in configuring the Cafeteria Management System (CMS).
\\ \\
{\em Please note that 'CMS' will be referred to in the rest of this document as an abbreviation for the Cafeteria Management System}


%%------------------------SETTING UP CMS ----------------------------
\subsection{Setting up CMS}
For the programmer, who will maintain the system: \\
Before starting the system, an email account has to be set up to facilitate communication between the system and end users. The details of this account can be configured in the following config file:
\begin{verbatim}
	~/Cafeteria_Management_System/config/env/production.js
\end{verbatim} 
{\em (The document can be opened in any text editor or IDE  such as NetBeans, WebStorm or atom - just to name a few )}
\\ \\
Under the section 'Mailer', the following fields should be specified:\\
\begin{itemize}
\item MAILER\textunderscore FROM: A name indicating the sender of mail.
\item MAILER\textunderscore SERVICE\textunderscore PROVIDER: The service provider of the email account
\item MAILER\textunderscore EMAIL\textunderscore ID: The email ID of the account set up for CMS
\item MAILER\textunderscore PASSWORD: The password of the account set up for CMS
\end{itemize}


In a terminal/command prompt, navigate to the CMS directory and execute the 'npm install' command. This will install all the packages required to run the system: 


\begin{verbatim}
~/  Cafeteria_Management_System/ $npm install
\end{verbatim}

If all dependencies were installed successfully, then MongoDB can be started with the following command in a completely new terminal or command prompt:
\begin{verbatim}
~/$mongod --dbpath "directory"
\end{verbatim}

Where "directory"  is a path to the folder which Mongo will use as a working directory.\\ \\
{\em Remember that this command has to be executed in a separate terminal.}\\ \\
Below is an example of what the output should look like :

\begin{figure}[H]
  \centering
    \includegraphics[width=1.0\textwidth]{screenshots/MongoDB.png}
    \caption{MongoDB Terminal - Expected output when running MongoDB} 
\end{figure}

Once mongo has been started, the CMS server can be started with the following command:\\ \\

\begin{verbatim}
~/Cafeteria_Management_System/$ grunt
\end{verbatim}

\begin{figure}[H]
  \centering
    \includegraphics[width=1.0\textwidth]{screenshots/gruntOutput.png}
    \caption{Grunt - When grunt is running, output should be similar to this.} 
\end{figure}

Now the Server and the Database are running and we can get started with the rest of the setup process. \\
Note: Inside the browser one will run localhost:3000 to view system. \\ 

To terminate the server, the user can enter the Ctrl+C command in the "grunt" terminal. The user can also terminate the MongoDB service by executing the same command (Ctrl+C) in the MongoDB terminal.

%%-------------------------------------GETTING STARTED ----------------------------------------
\section{Getting Started}
Access to the Cafeteria Management System is through a standard web browser. Different types of users have access to different facets of the system. The system has a default super user account and an admin user account, where both of these users can assign different roles (cafeteria manager, cashier, etc.) to the users. They also have global access to the whole system. These users can then sign in to access the facet of the system they are authorised to.\\

%%--------------------------------ADMINISTRATIVE USERS -------------------------
\subsection{Administrative Users}
When the CMS is started initially with an empty database there will be no users in the database and this includes no administrative users.  \\
Thus, to generate the administrative users, on the first startup of the system, one should navigate to the sign in page and sign in with empty credentials. If the system is started for the first time with an empty database, on sign in with empty credentials (proceeding to submit the signin form without filling in username or password) administrative users will be created. \\

\textbf{The system will have a super user:} \\
Employee ID: SuperUser \\
Password: SuperUser \\

\textbf{And the system will also have an admin user:} \\
Employee ID: AdminUser \\
Password: AdminUser \\

\textbf{WARNING :} \\
Administrative users will have global access to the whole system, thus it is of utmost importance that the administrative users should be set up with the first start up of the system and their credentials should immediately be changed for security. \\ 

{\em* Note at all times there can only be 1 super user and 1 admin user - this is done for security purposes  } \\
 

%%-------------------------------- CREATING AN ACCOUNT -------------------------
\subsection{Creating an Account} 
Once the user has clicked the "Sign Up" option on the navigation pane, the user will be directed to the signup form, where the user should fill in their details. 

\begin{figure}[H]
  \centering
    \includegraphics[width=1.0\textwidth]{screenshots/signUp.PNG}
    \caption{Sign Up - red circled button should be clicked to sign up} 
\end{figure}

When the button is clicked, the CMS will direct the user to the signup page where the user can fill out all the details. Once the user have completed the form, the user will click submit and if the form is correctly filled in , the user will be notified upon success and will be signed up for the system. They will hence be redirected to the home page. The user will then use the password created and employee ID to log in to the system. If the information entered is not valid, a thorough error message will be displayed indicating what the problem is so that the user can rectify it.

\begin{figure}[H]
  \centering
    \includegraphics[width=1.0\textwidth]{screenshots/signUpPage.PNG}
    \caption{Sign Up Page - details to be filled in - all fields are required} 
\end{figure}

{\em Employee ID } will be assigned to users by their company - no Employee ID can be reused.\\
The {\em e-mail address} to receive notifications of when their orders are ready and to receive monthly financial bills. \\
The {\em spending} Limit is the maximum amount you may spend each month.\\

{\em Note that all the fields may be edited when logged in} \\

After signing up and creating a new account the user will automatically be logged in. 

%%-------------------------------- LOGGING IN  -------------------------
\subsection{Logging In}
To sign in, the user must click on the 'Sign In' tab on the navigation bar.

\begin{figure}[H]
  \centering
    \includegraphics[width=1.0\textwidth]{screenshots/signIn.png}
    \caption{Sign In - red circled button should be clicked to sign in} 
\end{figure}

Once the user clicks on the sign in tab, the CMS should direct to the sign in page :

\begin{figure}[H]
  \centering
    \includegraphics[width=1.0\textwidth]{screenshots/signIn.png}
    \caption{Sign In page - Type the appropriate information in the textboxes and click submit to sign in} 
\end{figure}
  
The user will fill in their password and Employee ID in the provided slots and click submit to proceed. If the information entered is valid, the user will be notified upon success and redirected to the home page, logged in on their personal account. If the information entered is not valid, a thorough error message will be displayed indicating what the problem is so that the user can rectify it.
If the user can not login due to forgetting his/her password they can click on the forget password link which will redirect to the forget password page:
 
\begin{figure}[H]
  \centering
    \includegraphics[width=1.0\textwidth]{screenshots/ForgotPass.png}
    \caption{Forgot password page} 
\end{figure}

The "Forgot your password?"option, which once clicks leads the user to a page where the user must enter their Employee ID. The user will then be notified that an email has been sent to their personal email account with further instructions on how to rectify the situation. The user will be sent a link to a page, in order to set a new password.    \\

The rest of the functionality will be described in the section below in detail  under the respective headings of how to navigate between pages to administrative settings to ordering an item and so forth. 
 
\begin{figure}[H]
  \centering
    \includegraphics[width=1.0\textwidth]{screenshots/emailSentForPass.png}
    \caption{After submitting the form - notified about email sent} 
\end{figure}

\begin{figure}[H]
  \centering
    \includegraphics[width=1.0\textwidth]{screenshots/newPassForPass.png}
    \caption{The url sent via email leads to this page - fill in the textboxes and new password is set} 
\end{figure}


%%-------------------------------------USING THE SYSTEM----------------------------------------
\section{Using The System} 

%%------------------------NAVIGATION PANE ----------------------
\subsection{The Navigation pane} 

%%------------------------NAVIGATION PANE USER  NOT LOGGED ON ----------------------
\subsubsection{The Navigation pane - once the user is not  logged on}
On the home page, the name of the canteen is displayed. The user will also see that the navigation pane is located at the top of the screen. The actions available on the pane are "Sign in", "Sign up", "Menu" and "On your plate". 
The Navigation pane will be displayed as follows if the user has not logged in:

\begin{figure}[H]
  \centering
    \includegraphics[width=1.0\textwidth]{screenshots/HomePage.PNG}
    \caption{Home Page - user not logged in } 
\end{figure}

The numbers in the images are described below:

\begin{enumerate}
\item The Home button - when clicking on this button it will always redirect to the home page
\item The Menu button - this button will direct user to the menu page
\item The on my plate button - this button will direct user to an orders page showing items you have currently ordered and the bill total
\item The Sign Up button - this button will direct to the signup page where a new account can be created
\item The Sign In button - this button will direct to a page where the user can signin and log into his/her account
\end{enumerate}

The user cannot proceed to order food if the user has not signed up and logged in. Hence, the first step a new user should take is signing up/ registering with the system. 
\\ \\
If a user has not signed in, the user will still be allowed, however, to view the menu, without ordering anything. However, the user will be able to add items to their plate and view them on the "On My Plate" page, but the order will not be sent to the system until the user signs in.


%%------------------------NAVIGATION PANE USER LOGGED ON  ----------------------
\subsubsection{The Navigation pane - once the user has logged on}
The user will now view a drop down menu with various options displayed on it. The following options will be displayed if the user is a normal user: "Edit Profile", "Profile", "Sign out". These pages will be discussed below.

\begin{figure}[H]
  \centering
    \includegraphics[width=1.0\textwidth]{screenshots/HomePage2.PNG}
    \caption{Home Page - user not logged in } 
\end{figure}


When a user is logged in as a normal user he/she will have the following view of the navigation pane as the above image.\\ below we describe what each of those buttons will do: \\

\begin{enumerate}
\item The Home button - when clicking on this button it will always redirect to the home page
\item The Menu button - this button will redirect to the menu page
\item The on my plate button - this button will redirect to an orders page showing items you have currently ordered and the bill total
\item The User button - this button will display the user's name and when clicked, will launch a dropdown menu with different options depending on what the role of the user is - the example used is just a normal user with the basic options. These are described below:
\item View Profile - this will redirect to the profile page of the respective user.
\item Edit Profile - this will redirect to the edit profile page where a user can change his/her current details and save it onto the CMS
\item Change Password - this will redirect to the change password page where a user can change his/her current password
\item Sign Out  - this will sign a user out of the CMS and redirect the user to the home page
\end{enumerate}

\textbf{Normal User}\\
A normal user will only have the options in the dropdown menu as displayed in the image above. If the user obtains another role, there will be extra settings displaying in the dropdown menu.\\

\textbf{Superuser or Admin User}\\
If the user is a superuser or an admin user the options "Admin Settings" and "Branding Settings" will also be displayed in the dropdown menu. The superuser and the admin user will hence be in control of assigning roles, changing employee ID's, setting the system wide spending limit, changing the canteen name and the cover image of the canteen.  \\

\textbf{Cafeteria Manager}\\
If the user is a cafeteria manager, the options "Manage Menu Items" and "Manage Inventory" will be displayed. Manage Inventory is where the stock additions and removals are kept track of. Manage Menu Items  is where the different menu meal items will be logged.\\

\textbf{Finance}\\
If user is a financial manager, the option "Finance" will be displayed. The financial manager will be able to search for employees and view their bills, to keep track of these.\\

\textbf{Cashier}\\
If the user is a cashier, the options "Placed Orders" will be displayed and it is here where the transactions will occur, such as marking whether orders are ready, and paid for. The cashier will also send notifications to the user, when the food is ready for the user to collect. 

%%------------------------MENU PAGE----------------------
\subsection{The "Menu" Page} 
This is where the user will be able to view the menu items and their prices. An item can be added to the user's plate by simply clicking the 'Add to Plate' button alongside each item. These can then be viewed on the 'On my Plate' page.
If a menu item is not in stock it will be written in red on the menu item that it is "out of stock " and there will be no  option to click the add to plate button since that item will not be available. 
\\
\begin{figure}[H]
  \centering
    \includegraphics[width=1.0\textwidth]{screenshots/searchCheese.png}
    \caption{The menu page from which a user can order food} 
\end{figure}
On the menu page there is also a breadcrumb which indicates the different meal categories to make the search more efficient. There is also a search bar on all these pages. 

\begin{figure}[H]
  \centering
    \includegraphics[width=1.0\textwidth]{screenshots/searchCheese.png}
    \caption{The menu page - Can be navigated via the search bar as illustrated} 
\end{figure}

\begin{figure}[H]
  \centering
    \includegraphics[width=1.0\textwidth]{screenshots/catMenu.png}
    \caption{The menu page - Can be navigated via the category breadcrumb as indicated to view sub menus} 
\end{figure}

\begin{figure}[H]
  \centering
    \includegraphics[width=1.0\textwidth]{screenshots/addToPlate.png}
    \caption{The menu page - order menu items by clicking the addToPlate button } 
\end{figure}

%%------------------------ON MY PLATE PAGE----------------------
\subsection{The "On my plate" Page} 
This page serves to indicate the current meal items that the user has on their plate. The user can also specify any preferences, if any, for each item on their plate, as well as the quantities of each item they wish to order. The total will dynamically increase/decrease accordingly when they add/remove items or change the quantities.

\begin{figure}[H]
  \centering
    \includegraphics[width=1.0\textwidth]{screenshots/viewOrder1.png}
    \caption{On my plate page - The meal you selected on the menu page is displayed here (Navigate here via the navigation bar)} 
\end{figure}
\begin{figure}[H]
  \centering
    \includegraphics[width=1.0\textwidth]{screenshots/viewOrder2.png}
    \caption{One can increase the quantity of items ordered by editing the quantity field indicated} 
\end{figure}
\begin{figure}[H]
  \centering
    \includegraphics[width=1.0\textwidth]{screenshots/viewOrder3.png}
    \caption{One can remove orders as well as specify preferences by clicking in the indicated areas} 
\end{figure}

%%----------------------EDIT PROFILE PAGE ----------------------
\subsection{The "Edit Profile" Page} 
The user will be presented with a similar form to that which they signed up with, however, the details that the user entered in the signup form will be present in these textboxes. The user can proceed to edit these here. Clicking the submit button will indicate whether changes have been saved or if errors have been made and how the user can correct these.

\begin{figure}[H]
  \centering
    \includegraphics[width=1.0\textwidth]{screenshots/editProfile.png}
    \caption{Edit profile page - Edit profile by typing into the text boxes and submitting for validation message and to save new information } 
\end{figure}

\begin{figure}[H]
  \centering
    \includegraphics[width=1.0\textwidth]{screenshots/limitExeeds.png}
    \caption{You must ensure your monthly spending limit is within the bounds of the maximum spending limit of the system, set by the super user} 
\end{figure}

%%------------------------PROFILE PAGE----------------------
\subsection{The "Profile" Page} 
This is where the user will be able to view their profile i.e. the details they entered when they signed up/ edited their profile. 

\begin{figure}[H]
  \centering
    \includegraphics[width=1.0\textwidth]{screenshots/viewProfile.png}
    \caption{The profile page - where you view your profile (via the orange navigation tab menu indicated)} 
\end{figure}

%%-----------------------CHANGE PASSWORD PAGE----------------------
\subsection{The "Change Password" Page} 
The user is presented with a form where the user will be asked to enter their old and new passwords to change their password. 

\begin{figure}[H]
  \centering
    \includegraphics[width=1.0\textwidth]{screenshots/changePassDontMatch.png}
    \caption{Can change password on this page - validation message will be displayed indicating if change was successful or not} 
\end{figure}

\subsection{Superuser: The "Administrative Settings" Page} 
At the top of the page there is a section labelled assign roles, where different admin roles will be assigned to different users. The super user simply has to type in an employee ID and select a role from the dropdown menu below. There is a section underneath that where the superuser can change the user ID of an employee. Self explanatory text boxes are provided for the superuser to fill in and the submit button will save the changes, unless an error occurs.  This page also consists of a section labelled "Change system limit" and it is here where the superuser can alter the limit of the system, i.e. the maximum value that a user can set their daily spending limits to. Hence a textbox is provided for the super user to type the new limit and save it.

\begin{figure}[H]
  \centering
    \includegraphics[width=1.0\textwidth]{screenshots/assignRole.png}
    \caption{The admin page - superuser can assign roles such as cashier to users} 
\end{figure}

\begin{figure}[H]
  \centering
    \includegraphics[width=1.0\textwidth]{screenshots/changeEmplid.png}
    \caption{The admin page - superuser can change the users' employee IDs} 
\end{figure}

\begin{figure}[H]
  \centering
    \includegraphics[width=1.0\textwidth]{screenshots/removeUser.png}
    \caption{The admin page - superuser can remove users from the system} 
\end{figure}

\begin{figure}[H]
  \centering
    \includegraphics[width=1.0\textwidth]{screenshots/setLimit.png}
    \caption{The admin page - superuser can set the monthly spending limit for the users} 
\end{figure}

\subsection{Superuser: The "Branding Settings" Page} 
There are two sections on this page. One where the user can change the canteen name, by merely typing in a new name over the old name in the allocated textbox, and another where the superuser can upload a new cover photo for the system.

\begin{figure}[H]
  \centering
    \includegraphics[width=1.0\textwidth]{screenshots/coverImage.png}
    \caption{The admin settings page - superuser can change the cover photo} 
\end{figure}

\begin{figure}[H]
  \centering
    \includegraphics[width=1.0\textwidth]{screenshots/canteenName.png}
    \caption{The admin settings page - superuser can change the canteen name}
\end{figure}

\subsection{Cafeteria Manager: The "Manage Inventory" Page}
This page is available to the cafeteria manager under the dropdown menu on the navigation pane. This is where the cafeteria manager adds inventory items to be used when an actual meal is stored in the menu in order to keep track of stock to note when a specific meal item is out of stock. These inventory items can be deleted, updated and searched for.

\begin{figure}[H]
  \centering
    \includegraphics[width=1.0\textwidth]{screenshots/addInv.png}
    \caption{The manage inventory page - Cafeteria Manager can add inventory items}
\end{figure}

\begin{figure}[H]
  \centering
    \includegraphics[width=1.0\textwidth]{screenshots/updateInv.png}
    \caption{The manage inventory page - Cafeteria Manager can update and delete inventory items}
\end{figure}

\subsection{Cafeteria Manager: The "Manage Menu" Page}
This is the page where the cafeteria manager adds menu items so that these can be displayed on the menu page found under the menu tab. The items can also be updated, deleted and searched for.

\begin{figure}[H]
  \centering
    \includegraphics[width=1.0\textwidth]{screenshots/addMenu.png}
    \caption{The manage menu page - Cafeteria Manager can add menu items}
\end{figure}

\begin{figure}[H]
  \centering
    \includegraphics[width=1.0\textwidth]{screenshots/searchMenuItem.png}
    \caption{The manage menu page - Cafeteria Manager can search for menu items}
\end{figure}

\begin{figure}[H]
  \centering
    \includegraphics[width=1.0\textwidth]{screenshots/updateMenu.png}
    \caption{The manage menu page - Cafeteria Manager can update menu items}
\end{figure}

\begin{figure}[H]
  \centering
    \includegraphics[width=1.0\textwidth]{screenshots/deleteHam.png}
    \caption{The manage menu page - Cafeteria Manager can delete menu items}
\end{figure}

\subsection{Cashier: The "Process Orders" Page}
This page is for use by the cashier. Orders that are "open" will be displayed on the page. Orders can be marked as ready, paid or collected. When the order is marked as ready, the button will disappear and the employee will be sent an email informing them to come collect their order. When the "Employee Paid" button is clicked, the cashier will choose whether it was a cash or credit purchase and the amount will be deducted accordingly from the user's account.

\begin{figure}[H]
  \centering
    \includegraphics[width=1.0\textwidth]{screenshots/cashier.png}
    \caption{The process orders page - The cashier is authorized to facilitate these transactions}
\end{figure}

\begin{figure}[H]
  \centering
    \includegraphics[width=1.0\textwidth]{screenshots/cashierReady.png}
    \caption{The process orders page - The cashier can mark orders as ready which will send a notification email to the user}
\end{figure}

\begin{figure}[H]
  \centering
    \includegraphics[width=1.0\textwidth]{screenshots/cashierCredit.png}
    \caption{The process orders page - The cashier can select the appropriate radio button as to whether the employee is paying with cash or credit.}
\end{figure}

\subsection{Financial Manager: The "View Employee Bills" Page}
There is a field labelled "Employee Id" and it is in here where a user will type in the employee ID 

\section{Troubleshooting}
\subsection{Problems with setting up the system}
If the system does not start up when you run the 'grunt' command, either of the following procedures can be followed:
\begin{itemize}
\item Ensure you have an active internet connection as the system requires an internet connection.
\item Ensure you have MongoDB running in a separate terminal. \\
	The line \begin{verbatim}
		Waiting for connections on port 27017
	\end{verbatim} should be displayed at the end of the Mongo terminal.
\item If the above does not solve the problem, the command npm update should be run from inside the CMS directory:
	\begin{verbatim}
		~/Cafeteria Management System$ npm update
	\end{verbatim}
\item If the problem persists the following commands should be run in order:
	\begin{verbatim}
		~/Cafeteria Management System$ bower install
	\end{verbatim} 
	\begin{verbatim}
		~/Cafeteria Management System$ npm install
	\end{verbatim} 
	\begin{verbatim}
		~/Cafeteria Management System$ npm update
	\end{verbatim}
\end{itemize}
\end{document}
