\documentclass[hidelinks, 12pt, oneside]{article}
\usepackage{bookmark}
\usepackage{graphicx}
\usepackage{hyperref}
\graphicspath{{images/}}
\usepackage[utf8]{inputenc}
\usepackage[english]{babel}
\begin{document}

 %titlepage
\thispagestyle{empty}
\begin{center}
\begin{minipage}{0.75\linewidth}
    \centering
    
 {\normalsize Project Tender\par}
 \vspace{1cm}
%Thesis title
    {\uppercase{\Large Project Name: Integrated Gynaecology Patient Information Management System.\par}}
   	{\Large Client: Prof. Leon C Snyman\par} 
    \vspace{1cm}

%Thesis title
    {\uppercase{\Huge Group Name\par}} 

%Author's name
    {\Large Rendani Dau 13381467\par}
    {\Large Elana Kuun u12029522\par}
    {\Large Semaka Malapane 13081129 \par}
    {\Large Antonia Michael 13014171\par}
    {\Large Isabel Nel 13070305\par}
    \vspace{1cm}
    
\end{minipage}
\end{center}
\clearpage

\tableofcontents
\newpage

\section{The Team}

\subsection{Antonia Michael}


\begin{figure}[h!]
  \centering
    \includegraphics[width=0.85\textwidth]{t} 
\end{figure}

\subsubsection{Interests}
My interests include working with people, helping people, reading, tennis, greek dancing, and spending time with friends and family. I am also passionate about my computer science degree and being one of few females in the degree. 
\subsubsection{Technical Skills}
My technical skills include: Java, C++, Html5, css, php, Bootstrap, Javascript, Semantic UI, Sql, php, python,  WebGL, Latex, Github, shell script, AngularJS, NodeJS, Microsoft Office, Npm, HandleBars Server, Mongoose, Gemfury, Mocha unit testing. 
\subsubsection{Past experiences relevant for project}
Programming \\
I have done an internship at a small IT company called Lepsta (Pty)Ltd, where I worked mainly on the front end of their new systems. I was mainly responsible to implement the front end functionality and gui of their email client app called Ridicle, using AngularJS, Python, Semantic UI, and using Brackets as the platform. I also worked on a similar email client app there.\\

I also assisted in implementing a CMS Website for small enterprise, NCC Solutions. In addition, for COS 301 I was chosen to be part of the Infrastructure Integration team to integrate the code given to us from the functional teams. We used NodeJS, RoboMongo, Mongoose, Nodemailer, Npm, Gemfury and HandleBars server for our integration. \\

Business Analysis \\
I also worked as a business analyst at Lepsta (Pty)Ltd, where my duties included firstly, requirements management - gathering, defining, analyzing and documenting business and functional requirements and designing a solution based on these requirements. Secondly, communicating requirements to programmers and other stakeholders ensuring that everyone understands what is required for the project to be a success. I also had to develop the business processes to provide a graphical flow of data, indicating the relationship between stakeholders and systems. I had to also design the test cases and implement the testing. Throughout the process I served as a communication channel, ensuring that all the teams knew exactly what was expected of them, and reported their progress and any concerns to me.\\

Project Management \\
At Lepsta (Pty)Ltd I also worked as a Project Manager. I developed a schedule for the project completion, allocating resources to the tasks. I also had to determine resources needed to complete the different phases of the project (i.e. time, money, technologies). I used a Gannt chart to allocate time frames for the different tasks in the project. I used a SCRUM board to keep track of the progress of the different tasks and to let the project needs see a visual representation of their tasks at hand. I had to keep regular communication with management to express the programmers' concerns and progress. It was also my job to develop team spirit and make sure that each member is contributing equally to the projects. \\
 
\subsubsection{Non technical strengths}
My non technical strengths include: \\

Firstly, I am a hard working, passionate and diligent student, who gives every task I do my full attention, and strive to only deliver good quality work. I work well with teams as I always consider each members view points and I dislike disputes, hence I always feel obliged to deliver more than my share in group projects in order to not let the large group of people that are depending on my efforts down. 
\\

Leadership skills - 
 \\1.) Head of Communications in The School of IT faculty House
 \\2.) Selected by the Department of Computer Science to be a 	Infrastructure Team Leader for the module COS 301 for the implementation of The Buzz System, 
 \\3.) Was a class Representative for the module, Imperative Programming.
   
Public speaking and presentation skills - Got chosen as the best overall paper and presentation from our presentation at the 2014 ACEIE Information Science Conference held at The University of Pretoria. Was then selected by UP to present research paper at the 2014 Information Ethics Conference held at The University of Zululand, competing with students from UJ, UP and other universities. We obrained 2nd place. 
\\
Social responsibility skills - Tutored for the Basic Computer Training course (6 sessions) for underprivileged female students eager to persue IT careers.\\ 
Organised and ran a Computer Training course for members of the community at UP Mamelodi Campus with a group of 4 other students - 40 hours of community work for JCP community development module in second year \\

Debutantes Year at school (2010) – Hosted various events/ programmes to raise funds for the Phehela Day and Night Centre and assisted with other similar charities.\\,

Teaching skills - In addition, the computer literacy courses I have taught, I am currently a Tutor in the Computer Science Department at the University of Pretoria.
Other skills: 
Communication skills, team work skills and people skills.

\subsubsection{What makes you want to do the project}
I would like to do the project because I have always found that most of my happiness lies in helping people and doing good for the community and this is where my passion lies.  Hence, when I found out that a project exists where we get the opportunity not only to code an app as expected, but to get the privilege to code an app for a previously disadvantaged hospital that is does not use computers to simplify and maintain its procedures, I immediately wrote it down as my personal first choice. This is in fact the project I would like to do the most because apart from the abovementioned, it is extremely important to maintain very accurate files with each patients different medical conditions and treatments, because people's lives are depending on it. If chosen to do this project I will not take this lightly because I will know that I am not merely programming to better my skills but making a huge impact and one that is long over due. 
\\
On the technical side, I thoroughly enjoyed working with databases in INF 214 (Information systems), as it was the first time in the degree that I knew we were learning a very relevant skill that we could directly apply in the industry. It would also be a very satisfying feeling to create a piece of software that the hospital can benefit from.
\\
I especially enjoyed the two modules, Netcentric computer systems and Database Management and Design in second year. In these modules we worked with systems, their databases and used Php, Sql, Mysql, Xxamp and Javascript, html and css, which I believe are skills that are going to come in very good use for this project. 
\\
Also, in the mini project for COS 301 (Software engineering) I worked with a MongoDB, writing schemas and using robomongo to view the database, which I also enjoyed. Hence, I feel that myself and my team will have the necessary skills needed for the project. 
\\
\subsection{Rendani Dau}
\subsubsection{Interests}
My interests include music, video games, hanging out with friends and any and everything related to Information Technology including but not limited to the latest technology trends and news and how technology can be used to better our everyday lives.
\subsubsection{Technical Skills}
My technical skills include: Programming skills in Java and C/C++. Web development technologies including HTML, JavaScript/jQuery, CSS, PHP, Bootstrap and NodeJS. Database programming using SQL (MySQL and Microsoft SQL) and Mongo DB. Programming in C\# including creating Windows Forms Applications, Advanced use of the .NET framework including linking databases (SQL Server, Microsoft Access, Oracle Database, etc.) to Windows Forms for CRUD functionality, creating Runtime Libraries (DLLs), creating client/server web applications with ASP.NET and creating reports with the SAP Crystal Reports plugin for Visual Studio. Also proficient with the Microsoft Office suite, Latex, XML and related technologies (XSL and DTD).
\subsubsection{Past experiences relevant for project}
I have successfully completed web development courses which will allow me to bring the necessary web development techniques to the team. I have also completed a Database design and management course.
\subsubsection{Non technical strengths}
My non-technical strengths include teamwork and communication skills which were honed by the COS301 Mini Project, thinking on my feet to solve problems and voicing my ideas.
\subsubsection{What makes you want to do the project}
This project will help the Kalafong hospital provide better care for their patients in a more efficient manner and can help with future research which might aide in saving lives and providing overall better health care. The project also delves into my core interest of using technology to better our everyday lives. Being a resident of Atteridgeville myself, it excites me that I have the opportunity to give back to the community that I am a part of in a way that will allow me to utilize my IT skills to the best of my ability to better the community.

\subsection{Elana Kuun}
\subsubsection{Interests}
Programming is a big interest, I enjoy solving complex problems and creating working systems that other people can use. My hobbies include photography, playing music, and running. I like to read and write and I enjoy hiking trails and the outdoors. I am also very interested in psychology. People fascinate me and enjoy it when I can help them or make a contribution to the community.
\subsubsection{Technical Skills}
Java, C++, HTML, CSS, PHP, JavaScript, XML, SQL, NodeJS, Mongoose, XQuery, JQuery
Some experience with databases, such as Neo4j, MongoDB, db4o, PostgreSQL.
\subsubsection{Past experiences relevant for project}
I have completed all my second and third year modules for my degree and believe this knowledge can assist me in this project. I particiapted in the COS 301 mini-project that prepared us for implementing big systems. I also completed two modules about databases and one specifically about web design which could be beneficial to this project as it relates to the required technologies. 
\subsubsection{Non technical strengths}
I am conscientious and work on a project until it is done. I take ownership of my work and deliver the best quality that I can. In a team I listen to the other team members and treat them with respect. I take everyone’s opinion into account, as well as voicing my own. When the situation calls for it I will take the lead. Communication is important and I do my best to keep other people up to date and follow the progress of the team. Being organised is important to me and I am always prepared. I always plan my work and stick to deadlines.
\subsubsection{What makes you want to do the project}
Completing this project will make a contribution to the community and it is something that is very important to me. I value research and the outcomes that it can have and I would value the opportunity to contribute to this in any way. Writing a system that can be used in a meaningful way is something that I always wanted to do. 
\\
I am also very interested in how this problem can be solved. This project is also very similar to the types of systems that I would want to develop in the future, it involves web development which I am interested in and would like to expand upon.

\subsection{Semaka Malapane}

\begin{figure}[h!]
  \centering
    \includegraphics[width=0.85\textwidth]{Semaka} 
\end{figure}

\subsubsection{Interests}

My interests are playing sudoku, cards, puzzles, cooking, reading, shopping and spending time with friends and family.

\subsubsection{Technical Skills}

My technical skills include: C++, Java, CSS, PHP, HTML, JavaScript, NodeJS, nodeunit, Mongoose, WebGL, Microsoft Office, SQL and XML.
 
\subsubsection{Past experiences relevant for project}

I have successfully completed COS 216 (Netcentric Computer Systems)  which taught me a the required web development skills required for this project. 
I have also successfully completed database management and design. Not only did I enjoy the module but it taught me the required skills - SQL and databases - for this project. 
I believe the other programming modules I have done in the past two and half years will help with making this system usable.

\subsubsection{Non technical strengths}

My non technical strengths include: 

Team work and communication skills - The Software Engineering course offered us a mini project at the beginning of the year where we had to work in multiple different teams with different kinds of people. This gave me the opportunity to learn to work in a team setting and to communicate appropriately.
 
Public speaking and presentation skills - I participated in public speaking in 2011 and 2012. This taught me not only good speech writing but valuable presentation skills.
I also took part in a computer training course at the UP Mamelodi Campus in 2014 - this gave me the opportunity to better my public speaking and presentation skills.

Social responsibility skills - I tutored a Basic Computer Training course (6 sessions) for underprivileged female students.
I organised and ran a Computer Training course for members of the community at UP Mamelodi Campus with a group of 4 other students - 40 hours of community work for JCP community development module in second year.

\subsubsection{What makes you want to do the project}

\subsection{Isabel Nel}

\begin{figure}[h!]
\centering
\includegraphics[width=50mm]{IsabelNel}
\end{figure}

\subsubsection{Interests}
During my past two years of studying Computer Science, I have developed a
passion for software development. Finding creative solutions to problems in
the form of software is something that definitely falls in my interests. I am
also fond of the outdoors, helping others and baking.

\subsubsection{Technical Skills}
My thecnical skills include: C++, Java, CSS, PHP, JavaScript, XML, HTML,
SQL, WebGL, Micrisoft Office and NodeJS.

\subsubsection{Past experiences relevant for project}
II have successfully completed my foundational programming modules and
more specialized modules at the University of Pretoria, thus I am capable
of producing adequate and usable software. I also participated in the mini-
project of COS 301, which prepared us for projects such as the implementation of a integrated information management system. I also took Biology as an elective in my first year of studies and thus I understand  the importance of documenting medical information.

\subsubsection{Non technical strengths}
I believe I am a good team player, with that I mean that I am a supportive
person when it comes to team work, I respect my other team members and
I am good in sharing my ideas and listening to other's ideas. I can take on
a leadership position if need be, such as the position I was placed in for the
mini-project of COS 301 as a middle level team lead.

\subsubsection{What makes you want to do the project}
I understand that medical information gathered from research is what can save lives in the future. With this knowledge and the age we are living in where technology is all around us, I do believe that it can make life easier for all medical personnel and it might even   save lives to get easy access to desired information. This is why I am excited about this project to assist in  the implementation of an integrated information management system  for the Kalafong Provincial Tertiary Hospital.

\section{The Plan of Action}

 
 
\subsection{The development methodology we will use:}

1.) Functional requirements -  We will first formulate the use cases from the business rules given. From that we will create the services contracts and the use case diagrams (including the use case diagram for the whole system as individual components). Activity diagrams will then be drawn. Lastly, we will create the domain model for the system.

2.) Software architecture documentation - We will first devise the architectural responsibilties for the system. We will then formulate the quality requirements, with reasons that each requirement was chosen, strategies to achieve the requirement and patterns to achieve the strategies. After which the Integration requirements will be drafted. Here the integration channels, access channels, protocols and API specifications will be explained. The Architecture contraints will then be formed, including the reference architecture, technologies, and operating systems that will be used. Lastly, we will decide on which architectural patterns best fit the requirements. 

3.) Implementation phase - The use cases will be modularized and developed independantly of eachother. After thorough unit testing of the individual modules, these will be integrated into the whole system. 

4.) Testing and verification - seperate unit testing of the individual modules will be done to check if the services contracts have been followed. Automated integration testing will also be done to check that the modules only integrate with the system if their services contract has been met. 

5.) Documentation - An instruction manual with screenshots will be created as well as additional documentaion explaining the structure of the system.  

\subsection{How will you keep the client informed about the status of the project:}

We will firstly email the client at the end of each  week, stating the group's progress in the week, we will also set up a monthly meeting with the client to demo and get feedback. This can become a weekly meeting if required. However, we will email the client whenever we have questions or concerns.

\subsection{Any initial ideas we have around solving some of the technical challenges:}

\subsection{Technologies the team intends to use:} 

At this point the following technologies have been considered, but this is subject to more research and contact with the client. 

-A server provided by the CS department at UP that we have access to. 

-Sql to create and manage the databases required for access numbers the students and doctors and other databases required.

-Github for version control 
 

\subsection{What the client will receive from us at the end of the project:}

A flexible, pluggable, fully functional software application  that will be maintainable, with detailed supporting documentation and an instruction manual.

\end{document}
